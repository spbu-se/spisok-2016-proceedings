\documentclass{spisok-article}

\usepackage{textcomp}
\usepackage{euscript}

% --------
   \def\quaabove{\raise3pt\hbox {${{\lower2pt  \hbox {${\scriptscriptstyle {\rm ?}}$}}}  \atop = $}}
   \def\bull{\vrule height.9ex width.8ex depth-.1ex}%square bullet
   \def\defabove{\raise3pt\hbox {${{\lower2pt  \hbox {${\scriptscriptstyle {\rm def}}$}}}  \atop = $}}
   \def\aa{\hbox{\rm \bf a}}
   \def\bb{\hbox{\rm \bf b}}
   \def\cc{\hbox{\rm \bf c}}
   \def\dd{\hbox{\rm \bf d}}
   \def\ee{\hbox{\rm \bf e}}
   \def\ff{\hbox{\rm \bf f}}
   \def\gg{\hbox{\rm \bf g}}
   \def\hh{\hbox{\rm \bf h}}
   \def\ll{\hbox{\rm \bf l}}

   \def\xx{\hbox{\rm \bf x}}
   \def\yy{\hbox{\rm \bf y}}
   \def\zz{\hbox{\rm \bf z}}
   \def\ww{\hbox{\rm \bf w}}

   \def\uu{\hbox{\rm \bf u}}
   \def\vv{\hbox{\rm \bf v}}

   \def\zero{\hbox{\rm \bf 0}}

   \def\laa{\hbox{\rm \bf A}}
   \def\lbb{\hbox{\rm \bf B}}
   \def\ldd{\hbox{\rm \bf D}}
   \def\lff{\hbox{\rm \bf F}}
   \def\loo{\hbox{\rm \bf O}}
   \def\lcc{\hbox{\rm \bf C}}

   \def\slaa{\hbox{\footnotesize \rm \bf A}}

   \def\oms{\raise3pt\hbox {${{\lower2pt  \hbox {${\scriptscriptstyle (S)}$}}}\atop \displaystyle{\omega}$}}
   \def\dsum{\raise3pt\hbox {${{\lower2pt  \hbox {${ .}$}}}\atop \displaystyle{+}$}}

   \font\eufrak=eufm10

% --------


\title{Гладкость пространств сплайнов второго порядка\thanks{Работа частично поддержана грантом РФФИ 15-01-08847}}
\author{Ю.$\,$К.$\,$Демьянович, Санкт-Петербургский государственный университет, y.demjanovich@spbu.ru}

\begin{document}

\maketitle

\begin{abstract}
      Известно, что среди полиномиальных сплайнов
       $B$-сплайны третьей степени однозначно характеризуются носителем
      из четырех  соседних сеточных промежутков  и
      принадлежностью к пространству $C^2$ (см.[1-4]), однако
       в некоторых случаях  компьютерная реализация
      неполиномиальных сплайнов более эффективна (см. [5-6]).
        Минимальные (вообще говоря, неполиномиальные)
      сплайны второго порядка на неравномерной сетке построены в
      работах [7-8].

      Цель данной работы построить  минимальные  неполиномиальные
      сплайны третьего порядка на неравномерной сетке, сформулировать
      необходимые и достатчные условия непрерывности этих сплайнов и их
      производных в узлах сетки, а также дать средства их
      эффективного построения.  Для достижения поставленной
      цели разработан новый подход, ибо аппарат векторной алгебры в ${\mathbb
      R}^3$, существенно использовавшийся ранее (см. [7-8]),
      для рассматриваемого здесь  случая  оказался непригодным.
\end{abstract}

\section{Введение}

    Сплайны известны достаточно давно: это понятие появилось в
    1946 году в работах Шонберга (I.J.Schoenberg). Полиномиальные
    сплайны активно используются в вычислениях при аппроксимации и
    интерполяции  функций, при решении задач математической физики
    (они сродни конечно-элементным аппроксимациям); наконец,
    сплайны --- мощный аппарат сжатия числовых информационных потоков
     (как с частичным, так и с полным воссстановлением
     информации). Различные подходы к построению сплайнов, а
     также различные виды сплайнов рассматривались в ряде
     фундаментальных работ (см. [1] -- [7]). Новые идеи
     использования сплайнов появились в
     связи со всплесковыми (вэйвлетными) разложениями.
     Сплайн-всплесковые разложения характеризуются простотой,
     устойчивостью, позволяют достичь асимптотически оптимального
     порядка аппроксимации (в отношении $N$-поперечника стандартных
     комактов), локальностью и рядом других преимуществ, связанных
     с реализацией вычислений на компьютере. Эти свойства
     возникают благодаря тому, что исходными соотношениями в этих
     построениях служат аппроксимационные соотношения.
     Получающиеся в результате пространства гладких сплайнов
     оказываются вложенными на вложенных сетках, так что
     проектирование объемлющего пространства сплайнов на вложенное
     пространство
     порождает сплайн-всплесковое разложение. Такое разложение
     позволяет исходный информационный числовой поток представить
     в виде двух потоков: основного и всплескового (вэйвлетного).
     В результате вместо исходного потока передается менее плотный
     основной поток, а всплесковый передается лишь в случае
     необходимости полного восстановления исходного потока.
     Сплайн-всплесковое разложение осуществимо, если пространства
     сплайнов вложены друг в друга на вложенных сетках,
     а такая ситуация возникает для  пространств гладких сплайнов,
     так что построение пространств гладких сплайнов актуально.

     Цель данной работы рассмотреть гладкие сплайны при достаточно
     произвольной генерирующей функции в аппроксимационных
     соотношениях. Здесь даны полные доказательтва непрерывной
     дифференцируемости сплайнов второго порядка (см. также [7]).

\section{Некоторые обозначения}

         Пусть  $\mathbb Z$ ---  множество всех целых чисел; введем
      обозначения:
      ${\mathbb Z}_+\defabove$ $ \{j\;|\; j\geq 0,\; j\in {\mathbb Z}\}$,
      ${\mathbb N}\defabove \{j\;|\; j>0,\; j\in {\mathbb Z}\}$,
      ${\mathbb R}^3$ -- пространство трехмерных
      вектор-столбцов $\aa,\bb,\cc,\ldots,\ff$; для нулевого
      вектора-столбца
       будем  использовать символ $\zero$. Компоненты векторов
      обозначаются квадратными скобками и нумеруются цифрами
      $0,1,2$; например, $\aa=([\aa]_0,[\aa]_1,[\aa]_2)^T$.
      К векторам применяются обычные матричные операции, так что для двух
      векторов $\aa,  \bb\in {\mathbb R}^3$ имеем
       $\aa^T\bb=\sum_{s=0}^2 [\aa]_s [\bb]_s$, в то время, как  $\aa\bb^T$ ---
       матрица с элементами $[\aa]_p [\bb]_q$ ($p$ и $q$ --- номера
       строки и столбца соответственно, $p,q=0,1,2$).
      Квадратную матрицу третьего  порядка с
       ве\-к\-тор-столбцами $\aa,\bb,\cc\in {\mathbb R}^3$
       обозначим  $(\aa,\bb,\cc)$, а ее определитель
       ${\rm det}(\aa,\bb,\cc)$.

      На конечном или бесконечном интервале $(\alpha,\beta)$
       вещественной оси ${\mathbb R}^1$ рассмотрим  сетку
      $ X\defabove \{x_j\}_{j\in {\mathbb Z}}$,
      $$ X:\;\ldots < x_{-1}< x_0< x_1<\ldots;\; $$
      $$\hbox{пусть}\;
      \alpha\defabove\lim_{j\to -\infty}x_j,\;\beta\defabove\lim_{j\to
      +\infty}x_j.
      \eqno(2.1)$$
   Введем обозначения  $M\defabove \cup_{j\in {\mathbb Z}}(x_j,x_{j+1})$,
  $S_j\defabove [x_j,x_{j+3}],\quad J_k\defabove
  \{k-2,k-1,k\},$
  $\quad k,j\in {\mathbb Z}.
  $

      При  $K_0\geq 1$ обозначим ${\cal X}(K_0,\alpha,\beta)$ класс
    локально квазиравномерных сеток на промежутке  $(\alpha,\beta)$,
    т.е. класс сеток вида (2.1) со свойством
     $ K_0^{-1}\leq (x_{j+1}-x_j)(x_j-x_{j-1})^{-1}\leq K_0\qquad
     \forall j\in {\mathbb Z}$.
    Введем характеристику $h_X$ мелкости сетки $X$, полагая\break
    $h_X\defabove\sup_{j\in {\mathbb Z}}\;(x_{j+1}-x_j).$

    Пусть  ${\mathbb X}(M)$ --- линейное пространство вещественнозначных
    функций, заданных на множестве $M$, а $C^S(\alpha,\beta)$ ---
    линейное пространство функций, непрерывных вместе  со всеми
    производными до порядка $S$ во внутренних точках интервала $(\alpha,\beta)$.

       Рассмотрим множество
      $\laa\defabove \{\aa_j\}_{j\in{\mathbb Z}}$
      векторов $\aa_j\in {\mathbb R}^3$; пусть    $A_j\defabove
   \Bigl(\aa_{j-2},\aa_{j-1},\aa_j\Bigr)$.
    Множество $\laa$  называется {\it полной цепочкой векторов}, если
   ${\rm det} A_j \neq 0$ для всех $ j\in {\mathbb Z}$.
   Совокупность всех полных цепочек будем обозначать $\mathbb A$.

\section{Пространства ($X,A,\varphi$)-сплайнов}

    Рассмотрим трех-компонентную вектор-функцию  $\varphi(t)$
    с компонентами из ${\mathbb X}(M)$. Если $\laa\in{\mathbb A}$, то функции
    $\omega_j(t),\; t\in M$,  $j\in  {\mathbb Z}$,
    однозначно определяются из условий
   $$\sum_{j'\in J_k} \aa_{j'}\omega_{j'}(t)\equiv \varphi(t)\quad
   \forall t\in (x_k,x_{k+1}), \; k\in {\mathbb Z};\;$$
   $$\omega_j(t)\equiv 0 \quad  \forall t\notin S_j\cap M.
   \eqno(3.1)$$

 По формулам Крамера  из (3.1) находим
   $$\omega_j(t)={{\rm det} \Bigl(\{\aa_{j'}\}_{j'\in J_k,j'\neq j}\;\parallel\;
   '^j\varphi(t)\Bigr)\over{\rm det} \Bigl(\{\aa_{j'}\}_{j'\in
   J_k}\Bigr)}\qquad  \forall t\in (x_k,x_{k+1}), \;\; \forall j\in J_k,
   \eqno(3.2)$$
   где значок $\;\parallel\;'^j$ означает, что определитель в числителе получается из
   определителя в знаменателе заменой столбца $\aa_j$ на столбец
   $\varphi(t)$ (с сохранением прежнего порядка следования столбцов).
   Из соотношения (3.1) ясно, что   ${\rm supp}\;\omega_j \subset
   S_j$. Более наглядны, но менее удобны для доказательств
    формулы, получающиеся развертыванием   соотношения (3.2):
   $$ \omega_{j}(t)={{\rm det}
   \bigl(\aa_{j-2},\aa_{j-1},\varphi(t)\bigr)\over
   {\rm det}\bigl(\aa_{j-2},\aa_{j-1},\aa_j\bigr)}
   \quad \hbox{при}\; t\in (x_j,x_{j+1}),
  $$
   $$ \omega_{j}(t)={{\rm det}
   \bigl(\aa_{j-1},\varphi(t),\aa_{j+1}\bigr)\over
   {\rm det}\bigl(\aa_{j-1},\aa_{j},\aa_{j+1}\bigr)}
   \quad \hbox{при}\; t\in (x_{j+1},x_{j+2}),
  $$
   $$ \omega_{j}(t)={{\rm det}
   \bigl(\varphi(t),\aa_{j+1},\aa_{j+2}\bigr)\over
   {\rm det}\bigl(\aa_{j},\aa_{j+1},\aa_{j+2}\bigr)}
   \quad \hbox{при}\; t\in (x_{j+2},x_{j+3}),
  $$

   В линейном пространстве ${\mathbb X}(M)$ содержится линейное пространство
   $$\widetilde  X_{(X,\slaa,\varphi)}
   \;\defabove \{\widetilde  u\;|\;
   \widetilde  u(t)\defabove
   \sum_{j\in  {\mathbb Z}}c_j \omega_j(t)\quad
    \forall t\in  M,\;\;\forall
   c_j\in {\mathbb R}^1 \},
   \eqno(3.3)$$
   называемое {\it пространством минимальных
   $(X,\laa,\varphi)$-сплайнов}\break
    (второго порядка); функции
   $\omega_j,\; j\in  {\mathbb Z}$, называются
   {\it  образующими} пространства $\widetilde  X_{(X,\slaa,\varphi)}$.

   %====================================

   Рассмотрим  возможность продолжения функции
   $\omega_j,\quad j\in {\mathbb Z}$,
    непрерывным образом на интервал $(\alpha,\beta)$.
    Множество всех функций, непрерывных на интервале $(\alpha,\beta)$,
    обозначим $C(\alpha,\beta)$; для любого натурального числа $S$
     введем также обозначение\break
   $C^S(\alpha,\beta)\defabove \{u \;|\; u^{(i)}\in C(\alpha,\beta)
   \;\forall i=0,1,2,\ldots,S\}$, полагая
   $C^0(\alpha,\beta)=C(\alpha,\beta)$ . Для вектор-функций
   $$\uu(t)\defabove
   ([\uu]_0(t),[\uu]_1(t),[\uu]_2(t))^T$$
   с компонентами $[\uu]_i(t),$ $i=0,1,2$, введем
   пространства\break
   $\lcc(\alpha,\beta)\defabove \{\uu \;|\; [\uu]_j\in C(\alpha,\beta)
   \;\forall j=0,1,2\}$,
   $\lcc^S(\alpha,\beta)\defabove \{\uu \;|\; [\uu]_j\in C^S(\alpha,\beta)
   \;\forall j=0,1,2\}$; как обычно, считаем
   $\lcc^0(\alpha,\beta)=\lcc(\alpha,\beta)$.

   {\bf Лемма 1.}{\it $\;$ Пусть   $\varphi\in\lcc(\alpha,\beta)$,
    $\laa$ --- полная цепочка
   векторов, и пусть фиксированы
   $k\in {\mathbb Z}$ и   $t_*\in [x_k,x_{k+1}]$.  Для того чтобы
   $$ \lim_{t\to t_*,\;\; t\in (x_k,x_{k+1})}\omega_j(t)=0,
   \eqno(3.4)$$
   необходимо и достаточно, чтобы
   $${\rm det} \Bigl(\{\aa_{j'}\}_{j'\in J_k,j'\neq j}\;\parallel\;^{'j}
    \varphi(t_*)\Bigr)=0.
   \eqno(3.5)$$
       }

    {\bf Доказательство.} При $t\in (x_k,x_{k+1})$ функция
    $\omega_j(t)$ имеет вид (3.2), так что учитывая непрерывность
    вектор-функции $\varphi(t)$ на отрезке $[x_k,x_{k+1}]\in
    (\alpha,\beta)$, видим, что условия  (3.4) и (3.5)
    эквивалетны. \bull

   {\bf Лемма 2.}{\it $\;$ Пусть   $\varphi\in\lcc(\alpha,\beta)$,
   $\laa$ --- полная цепочка векторов,
   и в узле  $x_k$ выполнены условия
    $$   \lim_{t\to x_k-0}
    \omega_{k-2}(t)=0,\qquad \qquad \lim_{t\to x_k+0}
     \omega_{k}(t)=0.
   \eqno(3.6)$$
   Тогда справедливо соотношение
       $$
          \lim_{t\to x_k-0}\omega_j(t)=
        \lim_{t\to x_k+0}\omega_j(t)
        \hbox{   при}\;\; j\in \{k-2,k-1\}.
   \eqno(3.7)$$
    }

    {\bf Доказательство.} $\;$ Заменяя $k$ на $k-1$ в соотношении (3.1),
     имеем
   $$\sum_{j\in J_{k-1}} \aa_j\omega_j(t)\equiv \varphi(t)\qquad
   \forall t\in (x_{k-1},x_k),
   $$
   откуда в пределе при $t\to x_k-0$
    ввиду первого из предположений (3.6) получаем
    $$\sum_{j\in \{k-2,k-1\}} \aa_j
    \lim_{t\to x_k-0}\omega_{j}(t)
    = \varphi(x_k).
   \eqno(3.8)$$
   Аналогично из (3.1) и второго соотношения в (3.6) находим
    $$\sum_{j\in \{k-2,k-1\}} \aa_j
    \lim_{t\to x_k+0}\omega_j(t)
   = \varphi(x_k).
   \eqno(3.9)$$
   Поскольку векторы $ \{\aa_{j}\;|\;k-2,k-1\}$ линейно независимы, то
   из тождеств (3.8) -- (3.9) следуют соотношения (3.7). \bull

   {\bf Теорема 1.}{\it $\;$ Пусть
     $\varphi\in\lcc(\alpha,\beta)$, $\laa$ --- полная цепочка
     векторов. Для того, чтобы   функции
      $\omega_j(t) \quad (\forall j\in {\mathbb Z})$ могли быть
   продолжены до функций, непрерывных на интервале
   $(\alpha,\beta)$ необходимо
   и достаточно, чтобы предельные значения  функций
   $\omega_j(t)$ $\forall j\in {\mathbb Z}$ на границе носителя каждой
   из них были равны нулю.
    }

    {\bf Доказательство.} $\;$ Необходимость очевидна. Докажем
    достаточность. Благодаря непрерывности вектор-функции
    $\varphi(t)$ достаточно исследовать
    непрерывность функций $\omega_j(t) $
    в узлах сетки $ X$.  Если узел $x_k$ находится на границе множества
    $  S_j$, то непрерывность  $\omega_j$ в этой
    точке вытекает из условия доказываемой теоремы. Если же  узел $x_k$
    лежит внутри этого множества, то  выполнены условия леммы 2, и,
    следовательно,   справедливо соотношение (3.7). \bull

   {\bf Теорема 2.}{\it $\;$ Пусть
     $\varphi\in\lcc(\alpha,\beta)$, $\laa$ --- полная цепочка
     векторов. Для того, чтобы   функции
      $\omega_j(t) \quad (\forall j\in {\mathbb Z})$ могли быть
   продолжены до функций, непрерывных на интервале
   $(\alpha,\beta)$, необходимо
   и достаточно, чтобы были выполнены соотношения
   $${\rm det}
   \bigl(\aa_{k-2},\aa_{k-1},\varphi(x_k)\bigr)=0\qquad
   \forall k\in {\mathbb Z}.
   \eqno(3.10)$$
    }

    {\bf Доказательство.} $\;$ Для доказательства достаточно
   показать, что обращение в нуль  предельных значений  функций
   $\omega_j(t) \quad (\forall j\in {\mathbb Z})$ на границе
    носителя каждой
   из них эквивалентно соотношениям (3.10).

   Записывая формулу (3.2) для $k=j$, имеем
   $$\omega_j(t)={{\rm det}
   \Bigl(\{\aa_{j'}\}_{j'\in J_j,j'\neq j}\;\parallel\;
   '^j\varphi(t)\Bigr)\over{\rm det} \Bigl(\{\aa_{j'}\}_{j'\in
   J_j}\Bigr)}\qquad  \forall t\in (x_j,x_{j+1}),
   $$
   или (что то же самое)
   $$ \omega_{j}(t)={{\rm det}
   \bigl(\aa_{j-2},\aa_{j-1},\varphi(t)\bigr)\over
   {\rm det}\bigl(\aa_{j-2},\aa_{j-1},\aa_j\bigr)}
   \quad \hbox{при}\; t\in (x_j,x_{j+1}).
  $$
    Следовательно,
   $$ \omega_{j}(x_j+0)={{\rm det}
   \bigl(\aa_{j-2},\aa_{j-1},\varphi(x_j)\bigr)\over
   {\rm det}\bigl(\aa_{j-2},\aa_{j-1},\aa_j\bigr)}
  $$
   так что обращение в нуль на левом конце носителя эквивалентно
   равенству
   $${\rm det}
   \bigl(\aa_{j-2},\aa_{j-1},\varphi(x_j)\bigr)=0,
   $$
    что ввиду произвольности $j\in {\mathbb Z}$ совпадает с
    формулой (3.10).

     Аналогичным образом применение соотношения (3.2) для $k=j+2$
     дает
   $$ \omega_{j}(t)={{\rm det}
   \bigl(\varphi(t),\aa_{j+1},\aa_{j+2}\bigr)\over
   {\rm det}\bigl(\aa_{j},\aa_{j+1},\aa_{j+2}\bigr)},
   \quad \hbox{при}\; t\in (x_{j+2},x_{j+3}),
  $$
   откуда
   $$ \omega_{j}(x_{j+3}-0)={{\rm det}
   \bigl(\varphi(x_{j+3}),\aa_{j+1},\aa_{j+2}\bigr)\over
   {\rm det}\bigl(\aa_{j},\aa_{j+1},\aa_{j+2}\bigr)},
  $$
   так что обращение в нуль на правом конце носителя эквивалентно
   равенству
   $${\rm det}
  \bigl(\varphi(x_{j+3}),\aa_{j+1},\aa_{j+2}\bigr)=0,
   $$
    что совпадает с
    формулой (3.10), если принять во внимание произвольность
    $j\in {\mathbb Z}$ и положить $j=k-3$.

    Теорема полностью доказана. \bull

    {\it Замечание 1. }{$\;$ Анализируя доказательство теоремы нетрудно
    заметить, что для возможности непрерывного продолжения функций
     $\omega_j(t) \quad (\forall j\in {\mathbb Z})$
    на интервал $(\alpha,\beta)$ необходимо
   и достаточно, чтобы выполнялось равенство
   нулю предельных значений функций на левом конце носителя
   (или  на правом конце носителя).
   }

   В дальнейшем для вектор-функции
   $\varphi\in \lcc^S(\alpha,\beta)$ введем обозначение
   $$\varphi_k\defabove \varphi(x_k),\qquad
   \varphi_k^{(i)}\defabove \varphi^{(i)}(x_k),\quad
   i=0,1,\ldots,S,\quad k\in {\mathbb Z}.$$

   {\bf Теорема 3.}{\it $\;$  Пусть  $S\in {\mathbb Z}_+$,
   $\varphi\in \lcc^S(\alpha,\beta)$,  $\laa\in{\mathbb A}$.
   Для того, чтобы производные  $\omega_j^{(S)}(t)$
   функций $\omega_j(t)$, $t\in M$, $\forall j\in {\mathbb Z}$, могли быть
   продолжены до функций, непрерывных на интервале $(\alpha,\beta)$,
   необходимо и достаточно, чтобы
   $${\rm det}
   \bigl(\aa_{k-2},\aa_{k-1},\varphi_k^{(S)}\bigr)=0\qquad
   \forall k\in {\mathbb Z}.
   \eqno(3.11)$$
    }

   {\bf Доказательство.} $\;$ Для доказательства достаточно
   продифференцировать соотношения (3.2) и применить рассуждения,
   аналогичные тем, которые были использованы при доказательстве
   предыдущей теоремы. \bull


   {\bf Следствие 1.}{\it $\;$ Пусть
   $\varphi\in \lcc^1(\alpha,\beta)$,  $\laa\in{\mathbb A}$.
   Для того, чтобы все функции
    $\omega_j(t)$, $t\in M$,
   $\forall j\in {\mathbb Z}$, могли быть
   продолжены до функций класса  $C^1 (\alpha,\beta)$,
   необходимо и достаточно, чтобы выполнялись соотношения
   $${\rm det} \bigl(\aa_{k-2},\aa_{k-1},
   \varphi_k^{(S)}\bigr)=0\qquad S=0,1\quad
   \forall k\in {\mathbb Z}.
   \eqno(3.12)$$
    }

    Введем обозначения
    $$\bb_s^T\xx\equiv \det(\varphi_s,\varphi\,'_s,\xx),
    \eqno(3.13)$$
         $$\aa_j\defabove {\rm det}\begin{pmatrix}
   \varphi_{j+1}\;& \;\varphi\;'_{j+1} \cr
   \;& \; \;   &\; \cr
   \det(\varphi_{j+2},\varphi\,'_{j+2},\varphi_{j+1})\;& \;
   \det(\varphi_{j+2},\varphi\,'_{j+2},\varphi\,'_{j+1})\;\cr
   \;& \; \;   &\; \cr
    \end{pmatrix},
    \eqno(3.14)$$
    считая внешний определитель символическим (относительно первой
    строки).
    %\end{document}

    Пусть выполнено условие

    $(A)$\qquad
    $\varphi\in \lcc^2[\alpha,\beta]$, а  вронскиан
    ${\rm det}(\varphi,\varphi\,',\varphi\,'')(t)$
    отличен от нуля на отрезке  $[\alpha,\beta]$.

    {\bf Лемма 3.}{\it $\,$Если выполнено условие $(A)$, то при
     достаточно мелкой сетке из класса ${\mathbb X}(K_0,\alpha,\beta)$
    цепочка векторов  (3.14) полная.
     }

     Эта лемма устанавливается использованием формулы Тейлора.


       {\bf Теорема 4.}{\it $\;$ Пусть $\varphi\in \lcc^1[\alpha,\beta]$,
       и цепочка векторов  (3.14) полная. Тогда
       $$\omega_j\in C^1(\alpha,\beta)\qquad\forall j\in\mathbb Z.
       $$
    }

   {\bf Доказательство.} Из (3.13) -- (3.14) для любого $j\in\mathbb Z$ имеем числовой определитель
             $$\bb^T_{j+1}\aa_j= {\rm det}\begin{pmatrix}
   \det(\varphi_{j+1},\varphi\,'_{j+1},\varphi_{j+1})\;& \;
   \det(\varphi_{j+1},\varphi\,'_{j+1},\varphi\,'_{j+1}) \cr
   \;& \; \;   &\; \cr
   \det(\varphi_{j+2},\varphi\,'_{j+2},\varphi_{j+1})\;& \;
   \det(\varphi_{j+2},\varphi\,'_{j+2},\varphi\,'_{j+1})\;\cr
   \;& \; \;   &\; \cr
    \end{pmatrix},
   $$
   первая строка которого состоит из нулей, а также определитель
   $$\bb^T_{j+2}\aa_j= {\rm det}\begin{pmatrix}
   \det(\varphi_{j+2},\varphi\,'_{j+2},\varphi_{j+1})\;& \;
   \det(\varphi_{j+2},\varphi\,'_{j+2},\varphi\,'_{j+1}) \cr
   \;& \; \;   &\; \cr
   \det(\varphi_{j+2},\varphi\,'_{j+2},\varphi_{j+1})\;& \;
   \det(\varphi_{j+2},\varphi\,'_{j+2},\varphi\,'_{j+1})\;\cr
   \;& \; \;   &\; \cr
    \end{pmatrix},
   $$
    с двумя одинаковыми строками; таким образом, оба определителя
    равны нулю. Следовательно
    $$\bb^T_{j+1}\aa_j=0,\qquad \bb^T_{j+2}\aa_j=0\qquad\forall j\in\mathbb
    Z.
    \eqno(3.15)$$
    Обозначая $\EuScript L\{\aa,\bb\}$ плоскость, проходящую через
    векторы $\aa$ и $\bb$, из (3.15) имеем
    $$\EuScript L\{\bb_{j+1},\bb_{j+2}\}\perp \aa_j\qquad\forall j\in\mathbb
    Z.
    \eqno(3.16)    $$
    Заменяя здесь $j$ на $j-1$, находим
    $$\EuScript L\{\bb_{j},\bb_{j+1}\}\perp \aa_{j-1}\qquad\forall j\in\mathbb
     Z.
    \eqno(3.17)   $$
    Из (3.16) -- (3.17) получаем
   $$\EuScript L\{\aa_{j-1},\aa_{j}\}\perp \bb_{j+1}\qquad\forall j\in\mathbb
      Z.
    \eqno(3.18)  $$
    С другой стороны, из (3.13) видно, что
    $$\EuScript L\{\varphi_{j+1},\varphi\,'_{j+1}\}\perp \bb_{j+1}\qquad\forall j\in\mathbb
      Z.
    \eqno(3.19)  $$
    Формулы (3.18) и (3.19) показывают, что рссматриваемые
    плоскости совпадают
   $$\EuScript L\{\aa_{j-1},\aa_{j}\}=\EuScript L\{\varphi_{j+1},\varphi\,'_{j+1}\}
    \qquad\forall j\in\mathbb  Z.
    \eqno(3.20)  $$
    Из соотношения (3.20) следуют равенства (3.12); тем самым
    доказана непрерывная дифференцируемость функций $\omega_j$. \bull

     Таким образом, для построения минимальных сплайнов
    класса  $C^1$ в качестве функций $[\varphi]_0$, $[\varphi]_1$,
     $[\varphi]_2$ достаточно взять фундаментальную
     систему решений дифференциального уравнения вида
     $y\;''+b_1(t)y\;'+b_0(t)y=0$ с
     непрерывными коэффициентами на отрезке  $[\alpha,\beta]$.
      В частном случае уравнения
     $y\;''=0$ получаем хорошо известные $B$-сплайны второй
     степени (см., например, [2-4]).

\makeatletter\renewcommand{\refname}{\intl@references}\makeatother
\begin{thebibliography}{8}

  \item Алберг Дж., Нильсон Э., Уолш Дж. Теория сплайнов и ее приложения. М. 1972.
  316 с.

  \item Стечкин С.Б., Субботин Ю.Н. Сплайны в вычислительной
  математике. М., 1976. 248 с.

   \item  Завьялов Ю.С., Квасов Б.И., Мирошниченко В.Л. Методы
   сплайн-функций. М., 1980. 352 с.

  \item Малоземов В.Н., Певный А.Б. Полиномиальные сплайны. Л.,
    1986. 120 с.

  \item Buhmann M.D. Multiquadratic Prewavelets on Nonequally Spaced
     Knots in One Dimension. Math. of Comput., 1995. Vol. 64, № 212.
     P. 1611-1625.

  \item Davydov O., Nurnberger G. Interpolation by $C^1$ splines of
     degree $q\geq 4$ on triangulations. J. Comput. and Appl. Math.,
     2000.  Vol. 126. P. 159-183.

  \item {\it Бурова И.Г., Демьянович Ю.К.} О гладкости сплайнов
      Ж. Математическое моделирование. 2004, т.16, №12. С.40-43

\end{thebibliography}

\end{document}
