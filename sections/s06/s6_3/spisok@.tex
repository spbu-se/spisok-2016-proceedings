\documentclass{spisok-article}
     %\usepackage{amssymb}
     \usepackage{color}
     \usepackage{amsmath}
     \usepackage{amscd}
          \usepackage{textcomp}
          \usepackage{euscript}
          
\newtheorem{Lemma}{Лемма}


\def\bull{\vrule height.9ex width.8ex depth-.1ex}%square bullet

   %\def\defabove{\raise3pt\hbox {${{\lower2pt  \hbox {${\scriptscriptstyle {\rm def}}$}}}  \atop = $}}

\newcommand{\defabove}{\stackrel{\rm def}{=\kern-3.6pt=}}

\def\quaabove{\raise5pt\hbox {${{\lower2pt  \hbox {${\scriptstyle ?}$}}}  \atop \equiv $}}

\def\aa{\hbox{\rm \bf a}}
   \def\bb{\hbox{\rm \bf b}}
   \def\cc{\hbox{\rm \bf c}}
   \def\dd{\hbox{\rm \bf d}}
   \def\ee{\hbox{\rm \bf e}}
   \def\ff{\hbox{\rm \bf f}}
   \def\gg{\hbox{\rm \bf g}}
   \def\hh{\hbox{\rm \bf h}}
   \def\ll{\hbox{\rm \bf l}}
   \def\xx{\hbox{\rm \bf x}}
   \def\yy{\hbox{\rm \bf y}}
   \def\zz{\hbox{\rm \bf z}}
   \def\uu{\hbox{\rm \bf u}}
   \def\vv{\hbox{\rm \bf v}}
   \def\ww{\hbox{\rm \bf w}}


   \def\laa{\hbox{\rm \bf A}}
   \def\lbb{\hbox{\rm \bf B}}
   \def\ldd{\hbox{\rm \bf D}}
   \def\lnn{\hbox{\rm \bf N}}
   \def\slaa{\hbox{\footnotesize \rm \bf A}}
   \def\ssdd{\hbox{\footnotesize \rm \bf d}}
   \def\lrr{\hbox{\rm \bf R}}

   \def\lff{\hbox{\rm \bf F}}
   \def\loo{\hbox{\rm \bf O}}
   \def\lcc{\hbox{\rm \bf C}}
   \def\lxx{\hbox{\rm \bf X}}
   \def\lnn{\hbox{\rm \bf N}}

   \def\ezero{\raise3pt\hbox {${{\lower2pt  \hbox {${\scriptscriptstyle 0}$}}}\atop \displaystyle{\rm \bf e}$}}
   \def\eone{\raise3pt\hbox {${{\lower2pt  \hbox {${\scriptscriptstyle 1}$}}}\atop \displaystyle{\rm \bf e}$}}
   \def\etwo{\raise3pt\hbox {${{\lower2pt  \hbox {${\scriptscriptstyle 2}$}}}\atop \displaystyle{\rm \bf e}$}}


   \def\omm{\raise3pt\hbox {${{\lower2pt  \hbox {${\scriptscriptstyle m}$}}}\atop \displaystyle{\omega}$}}
   \def\omone{\raise3pt\hbox {${{\lower2pt  \hbox {${\scriptscriptstyle 1}$}}}\atop \displaystyle{\omega}$}}
   \def\omtwo{\raise3pt\hbox {${{\lower2pt  \hbox {${\scriptscriptstyle 2}$}}}\atop \displaystyle{\omega}$}}
   \def\omthree{\raise3pt\hbox {${{\lower2pt  \hbox {${\scriptscriptstyle 3}$}}}\atop \displaystyle{\omega}$}}

   \def\oms{\raise3pt\hbox {${{\lower2pt  \hbox {${\scriptscriptstyle (S)}$}}}\atop \displaystyle{\omega}$}}
   \def\dsum{\raise3pt\hbox {${{\lower2pt  \hbox {${ .}$}}}\atop \displaystyle{+}$}}

   \def\symb{\scriptscriptstyle {*}}
      \def\symbb{\scriptscriptstyle {\diamond}}

   \def\kzv{\hbox{\small \scriptsize  $K_*$}}
   \def\kzvpone{\hbox{\small \scriptsize  $K_*+1$}}
   \def\lkzv{\hbox{\small\footnotesize  $K_*$}}
   \def\lkzvpone{\hbox{\small\footnotesize  $K_*+1$}}


   \def\kmfive{\hbox{\small \scriptsize\tiny  $k-5$}}
   \def\kmfour{\hbox{\small \scriptsize\tiny  $k-4$}}
   \def\kmthree{\hbox{\small \scriptsize\tiny  $k-3$}}
   \def\kmtwo{\hbox{\small \scriptsize\tiny  $k-2$}}
   \def\kmone{\hbox{\small \scriptsize\tiny $k-1$}}
   \def\kmzero{\hbox{\small\scriptsize\tiny $k$}}
   \def\kpone{\hbox{\small \scriptsize\tiny  $k+1$}}
   \def\kptwo{\hbox{\small \scriptsize\tiny  $k+2$}}
   \def\kpthree{\hbox{\small \scriptsize\tiny  $k+3$}}
   \def\kpfour{\hbox{\small\tiny $k+4$}}
   \def\kpfive{\hbox{\small\tiny $k+5$}}

   \def\mtwo{\hbox{\small \scriptsize\tiny  $-2$}}
   \def\nmone{\hbox{\small \scriptsize\tiny  $N-1$}}
   \def\nmtwo{\hbox{\small \scriptsize\tiny  $N-2$}}

   \def\kprmtwo{\hbox{\small \scriptsize $k+r-2$}}
   \def\kprmone{\hbox{\small \scriptsize $k+r-1$}}
   \def\kprmzero{\hbox{\small \scriptsize $k+r$}}
   \def\kprpone{\hbox{\small \scriptsize $k+r+1$}}

   \def\tmtwo{\hbox{\small \scriptsize $\widehat{-2}$}}
   \def\tkmfive{\hbox{\small \scriptsize $\widehat{k-5}$}}
   \def\tkmfour{\hbox{\small \scriptsize $\widehat{k-4}$}}
   \def\tkmthree{\hbox{\small \scriptsize  $\widehat{k-3}$}}
   \def\tkmtwo{\hbox{\small \scriptsize  $\widehat{k-2}$}}
   \def\tkmone{\hbox{\small \scriptsize $\widehat{k-1}$}}
   \def\tkmzero{\hbox{\small\scriptsize $\widehat{\phantom{a}k\phantom{a}}$}}
   \def\tkpone{\hbox{\small \scriptsize  $\widehat{k+1}$}}
   \def\tkptwo{\hbox{\small \scriptsize  $\widehat{k+2}$}}
   \def\tkpthree{\hbox{\small \scriptsize  $\widehat{k+3}$}}
   \def\tkpfour{\hbox{\small \scriptsize $\widehat{k+4}$}}
   \def\tkpfive{\hbox{\small \scriptsize$\widehat{k+5}$}}

\title{Построение сетки адаптивного типа\thanks{Работа выполнена при финансовой поддержке РФФИ, грант № 15-01-08847}
}

\author{Сазонова Г. О., студент СПбГУ, galinasazo@gmail.com\\Демьянович Ю. К., профессор СПбГУ, yuri.demjanovich@gmail.com,}
}

\begin{document}

\maketitle

\section{Введение}
   В современном мире потоки информации имеют электронную форму.
   Как правило, это последовательность 0 и 1 внушительной длины
   ($10^{12} - 10^{16}$ символов). Такие объемы обрабатываются быстро
   только в случае, когда имеются достаточно большие компьютерные ресурсы
   (память, быстродействие и т.д.); поэтому актуален вопрос о сокращении
   объемов цифровой информации за счет отбрасывания несущественных
   ее составляющих.

   На первом месте среди средств разрешения данного вопроса
   несомненно находятся вейвлеты, что подтверждается большим числом
   приложений в различных технических и научных областях. Вейвлетное
   разложение рассматривается, как правило, на равномерной сетке
   (см., например, \cite{leb, ter}). Но за последнее время получили
   распространение сплайн-всплесковые  разложения, ассоциируемые с
   неравномерной сеткой (см. \cite{mall, dem}). Введение неравномерных
   сеток важно в случае нерегулярного поведения исходного потока: в
   областях медленного изменения упомянутого потока естественно
   использовать крупную сетку, а в областях быстрого изменения
   необходима мелкая сетка. При таком подходе возможно
   последовательное адаптивное укрупнение возникающих таким образом
   неравномерных сеток для получения всплескового пакета с заданной
   аппроксимацией исходного потока.
%\newpage
   \section{Сетка адаптивного типа}
    Использование цепочек вложенных пространств сплайнов позволяет
  строить вейвлетные разложения (декомпозицию и реконструкцию)
  в весьма общих условиях, в том числе, с использованием неравномерной
  сетки. Далее будет рассмотрена сетка адаптивного типа, которая
  зависит от заданного числового потока $f$ и положительного параметра
  $\varepsilon$.

  Пусть на интервале $(\alpha,\beta)$ рассматривается сетка
 $$\Xi:\quad
    \ldots<\xi_{-2}<\xi_{-1}<\xi_0<\xi_1<\xi_2\ldots, $$
    $$\lim_{i\to -\infty}\xi_i=\alpha,\quad \lim_{i\to
    +\infty}\xi_i=\beta.
            $$
    Множество функций $u(t)$, заданных на сетке $\Xi$, обозначим
    $C(\Xi)$. \linebreak Ясно, что  $C(\Xi)$ --- линейное пространство.

     Пусть $f\in C(\Xi)$, и для некоторой константы $c>0$
     справедливо соотношение
    $$ f(t)\geq
     c\quad\forall t\in\Xi.
    \eqno(1)    $$

    Если  $a\in\Xi$, то существует такое $i\in\mathbb Z$, что
    $a=\xi_i$. В этом случае обозначим $a^-\defabove\xi_{i-1} $,
    $a^+\defabove\xi_{i+1}$.

    Дальше предполагается, что
    $$a,b\in \Xi,\quad a^+<b^-,
     \eqno(2)   $$
     т.е. для некоторых $i,j\in\mathbb Z$,
    $i+2<j$, верны равенства $a=\xi_i$, $b=\xi_j$. При упомянутых
    $a$ и $b$ введем  обозначение
     $$\textlbrackdbl $a,b$\textrbrackdbl \defabove\{\xi_s\;|\;a\leq\xi_s\leq
     b,\; s\in\mathbb Z\},$$
      т.е.
      
      \textlbrackdbl$a,b$\textrbrackdbl$\defabove\{\xi_s\;|\;i\leq s\leq j,\;
      s\in\mathbb Z\}$. Множество
      \textlbrackdbl$a,b$\textrbrackdbl $\;$ будем называть сеточным
      отрезком.

      Рассмотрим линейное нормированное пространство
      $C$\textlbrackdbl$a,b$\textrbrackdbl$\;$ функций $u(t)$, заданных на
      сеточном отрезке \textlbrackdbl$a,b$\textrbrackdbl, где
      норма вводится соотношением
      $$\|u\|_{C\hbox{\textlbrackdbl$a,b$\textrbrackdbl}}\defabove
      \max_{t\in \hbox{\textlbrackdbl$a,b$\textrbrackdbl}}|u(t)|.
            $$
    Очевидно, что пространство $C\hbox{\textlbrackdbl$a,b$\textrbrackdbl}$  конечномерно.

   Пусть
  $$\varepsilon\in(\varepsilon^*,\varepsilon^{**}),
   \eqno(3)  $$
  где
   $$ \varepsilon^*\defabove\max_{\xi\in\hbox{\textlbrackdbl$a,b^-$\textrbrackdbl}}
  \max_{t\in \{\xi,\xi^+\}}f(t)(\xi^+-\xi),\quad
  \varepsilon^{**}\defabove (b-a)\|f\|_{C\hbox{\textlbrackdbl$a,b$\textrbrackdbl}}.
   \eqno(4)  $$

    Справедливо следующее утверждение.

    \begin{Lemma}
      \it $\,$ Если выполнены условия (1), (3), (4), то существуют и единственны натуральное число
   $K=K(f,\varepsilon,\Xi)$   и сетка
    $$\widetilde{X}=\widetilde{X}(f,\varepsilon,\Xi):\quad  a=\widetilde{x_0}
    <\widetilde{x_1}<\ldots< \widetilde {x_{K}}\leq \widetilde{x_{K+1}}=b
    \eqno(5)    $$
   такие, что
  $$\max_{t\in\hbox{\rm\textlbrackdbl}\widetilde{x_s},\;
        \widetilde{x_{s+1}}\hbox{\rm\textrbrackdbl}} f(t)
    (\widetilde{x_{s+1}}-\widetilde{x_s})\leq\varepsilon
    <\max_{t\in\hbox{\rm\textlbrackdbl}\widetilde{x_s},\;
        \widetilde{x^+_{s+1}}\hbox{\rm\textrbrackdbl}} f(t)
    (\widetilde{x^+_{s+1}}-\widetilde{x_s})
    \eqno(6)$$
    $$\forall s\in\{0,1,\ldots,K-1\},
    $$
    $$\max_{t\in\hbox{\rm\textlbrackdbl}\widetilde{x_K},\;b\hbox{\rm\textrbrackdbl}} f(t)
    (b-\widetilde{x_K})\leq\varepsilon,  \quad \widetilde{X}\subset \Xi.
    \eqno(7)$$
    \end{Lemma}

   {\bf Доказательство.}
    Проводится по индукции.
База индукции устанавливается следующим образом. Пусть
    переменная $\tau\in\Xi$ увеличивается от $a=\widetilde{x_0}$ до $b$;
    тогда ввиду предположения (1) функция
    $\phi_0(\tau)\defabove \max_{t\in\hbox{\rm\textlbrackdbl}\widetilde{x_0},\;
      \tau\hbox{\rm\textrbrackdbl}} f(t)(\tau-\widetilde{x_0})$
    является строго возрастающей и при изменении $\tau$ от
     $a=\widetilde{x_0}$ к $b$ функция $\phi_0(\tau)$  возрастает от $0$ до
    $\max_{t\in\hbox{\rm\textlbrackdbl}a,\;b\hbox{\rm\textrbrackdbl}} f(t)(b-a)$.
    Благодаря условию (3), (4) существует единственная точка $\tau_1\in
    \hbox{\rm\textlbrackdbl}a,\;b\hbox{\rm\textrbrackdbl}$
    такая, что
   $$\max_{t\in\hbox{\rm\textlbrackdbl}\widetilde{x_0},\;
         \tau_1\hbox{\rm\textrbrackdbl}} f(t)
    (\tau_1-\widetilde{x_0})\leq\varepsilon
    <\max_{t\in\hbox{\rm\textlbrackdbl}\widetilde{x_0},\;
        \tau^+_1\hbox{\rm\textrbrackdbl}} f(t)
    (\tau^+_1-\widetilde{x_0}).
    $$


    Положим $\widetilde{x_1}\defabove \tau_1$. База индукции установлена.

    Предположим, что узлы $\widetilde{x_1}$, $\dots$, $\widetilde{x_s}$ сетки
    $\widetilde X$ определены. Если $\widetilde{x_s}=b$, то полагаем $K\defabove s-1$.
    В этом случае построение сетки $\widetilde{X}(f,\varepsilon,\Xi)$ завершено.
    В противном случае $\widetilde x_s<b$,
    и построение сетки продолжается. Рассмотрим функцию
    $$\phi_s(\tau)\defabove\max_{t\in\hbox{\rm\textlbrackdbl}\widetilde{x_s},\;
      \tau\hbox{\rm\textrbrackdbl}} f(t)(\tau-\widetilde{x_s});$$
    она  является строго возрастающей: при изменении $\tau$ от
     $\widetilde x_s$ до $b$ функция $\phi_s(\tau)$  возрастает от $0$ до
    $m_s\defabove\max_{t\in\hbox{\rm\textlbrackdbl}\widetilde{x_s},\;
      b\hbox{\rm\textrbrackdbl}} f(t)(b-\widetilde{x_s})$.
   Заметим, что если  $\widetilde{x_s}=b^-$, то
   %\newpage
   $$m_s=\max_{t\in\{b^-,b\}}f(t)(b-b^-)\leq
   \max_{\xi\in\hbox{\textlbrackdbl$a,b^-$\textrbrackdbl}}
  \max_{t\in \{\xi,\xi^+\}}f(t)(\xi^+-\xi)=\varepsilon^*,
  $$
  и по предположению (3), (4) имеем $m_s\leq\varepsilon$.
   Во всех случаях, когда  $m_s\leq \varepsilon,$ полагаем
   $K\defabove s$ и  $\widetilde x_{s+1}=b$.
Рассмотрим случай, когда  $\varepsilon<m_s$. Из предыдущего следует,
    что $\widetilde x_s<b^-$. Пусть при
    некоторых  $p,q\in \mathbb Z$ справедливы соотношения
     $\widetilde x_s=\xi_p$ и
    $m_s= \max_{t\in\hbox{\rm\textlbrackdbl}\widetilde x_s,\;
      \xi_q\hbox{\rm\textrbrackdbl}} f(t)(\xi_q-\widetilde x_s)$.
      Очевидно, что $p<q$ (равенство
    $p=q$ дает $m_s=0$, что противоречит соотношению
    $\varepsilon<m_s$).

    Поскольку $0<\varepsilon<m_s$, а дискретная функция $\phi_s(\tau)$ принимает
    возрастающие значения от $0$   до $m_s$, то найдется такое
    $j$,  что $\xi_j\in\hbox{\rm\textlbrackdbl}\widetilde x_s,\;
      b^-\hbox{\rm\textrbrackdbl}$ и $\phi_s(\xi_j)\leq
      \varepsilon<\phi_s(\xi_{j+1})$. Последнее эквивалентно
      соотношению
        $$\max_{t\in\hbox{\rm\textlbrackdbl}\widetilde x_s,\;
        \xi_j\hbox{\rm\textrbrackdbl}} f(t)
    (\xi_j-\widetilde x_s)\leq\varepsilon
    <\max_{t\in\hbox{\rm\textlbrackdbl}\widetilde x_s,\;
        \xi_{j+1}\hbox{\rm\textrbrackdbl}} f(t)
    (\xi_{j+1}-\widetilde x_s).
    $$

    Положим $\widetilde x_{s+1}\defabove\xi_j$.
    Существование  точки  $\widetilde x_{s+1}$,
    удовлетворяющей соотношениям (6),
    установлено. Возможными ее значениями являются
    следующие узлы исходной сетки $\xi_{p+1}$, $\xi_{p+2}$,
    $\dots$,  $\xi_{q-1}$.  Единственность точки
     $\widetilde x_{s+1}$ следует из строгого возрастания функции
   $\phi_s(\tau)$.

      Итак, если $\varepsilon\geq m_s$, то
    полагаем $K\defabove s$ и $\widetilde x_{s+1}=b$; при этом
    выполнено соотношение (7).
      Если же $\varepsilon<m_s$, то найдется единственная точка
   $\tau_{s+1}<b$ так, что справедливо неравенство (6).
    Индукционный переход закончен.

    Лемма доказана.

      Сетку вида (5) со свойствами (6), (7) будем называть {\it сеткой
   адаптивного типа для дискретной функции $f$}.

    Очевидно, что целочисленная функция  $K(f,\varepsilon,\Xi)$ обладает
    свойством монотонности: если $\varepsilon\,'\leq\varepsilon\,''$,  то
   $K(f,\varepsilon\,',\Xi)\geq K(f,\varepsilon\,'',\Xi)$.

    Суммированием соотношений (6) получаем неравенство
      $$\sum_{s=0}^{K-1}\max_{t\in\hbox{\rm\textlbrackdbl}\widetilde x_s,\;
        \widetilde x_{s+1}\hbox{\rm\textrbrackdbl}} f(t)
    (\widetilde x_{s+1}-\widetilde x_s)\leq K\varepsilon<$$$$<
\sum_{s=0}^{K-1}\max_{t\in\hbox{\rm\textlbrackdbl}\widetilde x_s,\;
        \widetilde x^+_{s+1}\hbox{\rm\textrbrackdbl}} f(t)
    (\widetilde x^+_{s+1}-\widetilde x_s).
    $$
%\newpage
    \section{О построении сетки адаптивного типа}
        Приведем иллюстрацию рассуждений, которые были использованы при
        доказательстве леммы, для случая, когда сетка $\Xi$ равномерная
        с положительным шагом $h>0$, а ее узлы задаются формулой $\xi_j=jh$.
        Пусть
    $$a=\xi_0<\xi_1<\xi_2<\xi_3=b,
    $$
    так что рассматриваемый сеточный отрезок имеет вид
    $\hbox{\rm\textlbrackdbl}a,\;
      b\hbox{\rm\textrbrackdbl}\defabove
        \linebreak =\{0,h,2h,3h\}$.
    Таким образом, $a=0$, $b=3h$. Ясно, что при этом выполнено
    условие (2):    $a^+=h<b^-=2h$. В этом случае     имеем
    $$\varepsilon^*=\max\{f(0),f(h),f(2h),f(3h)\}h=
    \|f\|_{C\hbox{\textlbrackdbl$a,b$\textrbrackdbl}}h,\quad
    \varepsilon^{**}=3\varepsilon^*.
    \eqno(8)$$
    Условие (3) принимает вид
    $$\varepsilon^*<\varepsilon<3\varepsilon^*.
    \eqno(9)$$
    В доказательстве леммы 1 сначала рассматривается функция
    $\phi_0$. Здесь она принимает вид
    $\phi_0(\tau)=\max_{t\in\hbox{\rm\textlbrackdbl}0,\;\tau\hbox{\rm\textrbrackdbl}}
    f(t)\tau$,
    причем ее аргумент $\tau$ пробегает значения
    $0$, $h$, $2h$, $3h$, так что $\phi_0(0)=0$,
    $\phi_0(h)= \linebreak =h\max\{f(0),f(h)\}$,
    $\phi_0(3h)=2h\max\{f(0),f(h),f(2h)\}$,
    $\phi_0(3h)= $\linebreak $=3h\|f\|_{C\hbox{\textlbrackdbl$a,b$\textrbrackdbl}}$.

    Полагаем $\widetilde x_0\defabove a$,
    т.е. в нашем случае  $\widetilde x_0\defabove 0$.
    Первый шаг, который необходимо сделать --- найти  $\widetilde
    x_1$ так, чтобы выполнялось соотношение (6) при $s=0$,
    т.е. среди значений $\tau\in \{0,h,2h\}$ найти такое значение $\tau$, чтобы
     выполнялось соотношение
        $$\phi_0(\tau)\leq\varepsilon<\phi_0(\tau^+)
      $$
     или, что то же самое, соотношение
      $$\max_{t\in\hbox{\rm\textlbrackdbl}0,\;
        \tau\hbox{\rm\textrbrackdbl}} f(t)
    (\tau)\leq\varepsilon
    <\max_{t\in\hbox{\rm\textlbrackdbl}0,\;
        \tau+h\hbox{\rm\textrbrackdbl}} f(t)
    (\tau+h).
    $$
    При  $\tau=0$ это соотношение принимает вид
    $0\leq\varepsilon< \max \{f(0),f(h)\}h;$
    такое неравенство противоречит  условиям (8), (9), так
    что для $\tau$ возможен лишь один из двух вариантов:
    $$\tau=h,\quad \max \{f(0),f(h)\}h\leq\varepsilon< 2h\max    \{f(0),f(h),f(2h)\},
    \eqno(10)$$
    %\newpage
    $$\tau=2h,\quad 2h\max \{f(0),f(h),f(2h)\}\leq\varepsilon< 3h\|f\|_{C\hbox{\textlbrackdbl$a,b$\textrbrackdbl}}.
    \eqno(11)    $$

    Если верно соотношение  (10), то согласно (6)  следует положить
    $$\widetilde x_1\defabove h,
    \eqno(12)    $$
     и перейти к отысканию $\widetilde x_2$. Для этого рассмотрим
     функцию $\phi_1(\tau)= \linebreak =
    \max_{t\in\hbox{\rm\textlbrackdbl}\widetilde x_1,\;\tau\hbox{\rm\textrbrackdbl}}
    f(t)(\tau-\widetilde x_1)$.
    Заметим, что
    $$\phi_1(h)=0,\quad
    \phi_1(2h)=h\max_{t\in\hbox{\rm\textlbrackdbl}h,\;2h\hbox{\rm\textrbrackdbl}}
    f(t)=h\max\{f(h),f(2h)\},
    \eqno(13)    $$
    $$\phi_1(3h)=2h\max_{t\in\hbox{\rm\textlbrackdbl}h,\;3h\hbox{\rm\textrbrackdbl}}
    f(t)=2h\max\{f(h),f(2h),f(3h)\}.
    \eqno(14)    $$

    Требуется  найти такое $\tau\in\{h,2h\}$,
    которое удовлетворяет соотношению
    $$\phi_1(\tau)\leq\varepsilon<\phi_1(\tau^+),
    \eqno(15)    $$
    и положить $\widetilde x_2\defabove \tau$.

    При $\tau=h$ соотношение (15) принимает вид
    $$0=\phi_1(h)\leq \varepsilon<\phi_1(2h).
    \eqno(16)    $$
    Легко видеть,  что (16) противоречит условиям  (8), (9).
    Остается рассмотреть лишь случай $\tau=2h$; в этом случае
    (15) примет вид
   $\phi_1(2h)\leq\varepsilon<\phi_1(3h),
   $
    или, в соответствии с формулами (13), (14), вид
     $$h\max\{f(h),f(2h)\}\leq \varepsilon<2h\max\{f(h),f(2h),f(3h)\}.
    \eqno(17)    $$
     Итак, если неравенство (17) выполнено, то полагаем
     $\widetilde x_2\defabove 2h$,  $\widetilde x_3\defabove 3h$.
     Учитывая соотношение (12), видим, что в этом случае искомая
     сетка $\widetilde X$ построена; при
     этом $\widetilde X=\{0,h,2h,3h\}$ (т.е. узлы новой сетки совпадают
     с последовательными узлами исходной сетки $\xi_j=jh,
     \;j=\linebreak=0,1,2,3$).

       Если неравенство (17) не выполнено, то ввиду условий
        (8), (9) заведомо выполнено условие вида (7); в
    рассматриваемом случае оно принимает вид
    %\newpage
     $$2h\max\{f(h),f(2h),f(3h)\}\leq \varepsilon<
     3h\|f\|_{C\hbox{\textlbrackdbl$a,b$\textrbrackdbl}},
        $$
     и потому полагаем $\widetilde x_2\defabove 3h$.
     
     Полученная сетка  $\widetilde X=\{0,h,3h\}$.

     До сих пор рассматривалась ситуация, когда выполнено
      неравенство (10). Теперь предположим, что выполнено
      неравенство (11):
      $\varphi_0(2h)\leq \varepsilon <\varphi_0(3h).
      $
      В этом случае полагаем $\widetilde x_1 \defabove 2h$ и дальше остается лишь положить
      $\widetilde x_2\defabove 3h$. Таким образом, здесь сетка $\widetilde
      {X}=\{0,2h,3h\}$.

      \section{Заключение}
      Доказана лемма, позволяющая строить сетку адаптивного типа, которая зависит от заданного числового потока и некоторого положительного параметра $\varepsilon$. Рассмотрен пример построения адаптивной сетки для случая, когда исходная сетка равномерная с положительным шагом.

\newpage
\renewcommand\refname{Литература}
\begin{thebibliography}{8}

\bibitem{leb} Лебедев А. С., Лисейкин В. Д., Хакимзянов Г. С. Разработка
    методов построения адаптивных сеток // Вычислительные технологии. 2002.
    Т. 7, № 3. С. 29--43.
\bibitem{ter} Terekhov K., Vassilevski Yu. Two-phase water
    flooding simu\-la\-tions on dynamic adaptive octree grids with
    two-point nonlinear fluxes// Russian Journal of Numerical Analysis
    and Mathematical Modelling. 2013. Vol. 28, No 3. P. 267--288.

\bibitem{mall} Малла С. Вейвлеты в обработке сигналов. М.: Мир, 2005. 671 с.

\bibitem{dem} Демьянович Ю. К. Теория сплайн-всплесков. СПб.: Изд-во \linebreak С.-Петерб. ун-та, 2013. 526 с.

\end{thebibliography}

\end{document}
