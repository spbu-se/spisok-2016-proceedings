
\documentclass{spisok-article}
\usepackage[space]{cite}

\title{О способе построения двумерной адаптивной сетки для обработки изображений.
}

\author{
  Гусев Д. М.,
  аспирант кафедры параллельных алгоритмов СПбГУ,
  os.shad@gmail.com
}

\begin{document}

\maketitle

\begin{abstract}
Предложен алгоритм построения двумерной адаптивной треугольной сетки, которое позволяет учесть особенности двумерного потока данных. Разработано приложение строящее данную сетку для изображений.
\end{abstract}

\section{Введение}
При обработке изображений в основном используется два подхода --- построение равномерной двумерной сетки и применение некоторого алгоритма сжатия для ячейки сетки и преобразование двумерного потока в одномерный и применение алгоритма к нему. В первом случае изображение не оценивается и, таким образом, может теряться качество на резких изменениях потока или происходить лишняя работа алгоритма на константном потоке данных. Кроме того может проявляться блочная структура изображения. Во втором случае построение адаптивной сетки не несет особых проблем, однако теряется информация о топологии исходного потока (оценка изображения идет только по одной оси) и ведет к неоптимальной обработке изображения. Вариант решения этой проблемы предлагается в \cite{demyan2016}, в которой предлагается использовать вэйвлетные разложения курантова типа, основаные на триангуляции рассматриваемой области.

В данной работе предлагается алгоритм построения адаптивной двумерной сетки, с оценкой изменения двумерного потока по обеим осям. Данную сетку можно применять как для вейвлетной обработки изображения, так и для стандартных алгоритмов сжатия.

\section{Оценки погрешности интерполяции на симплексе триангуляции}
Пусть $T$~--- симплекс из триангуляции некоторой исходной области в $R^m$, $f$~--- функция $m$ переменных, непрерывная на $T$ вместе со всеми частными производными до порядка $n+ 1$ включительно, причем производные $n+ 1$ порядка по любым направлениям $\xi_1, \dots ,\xi_n$ ограничены на $T$ числом $M_{n+1}$:
$$\left \| D^{n+1}_{\xi_1, \dots ,\xi_n}f \right \|_T \le M_{n+1}. \eqno( 1 )$$
Рассмотрим задачу интерполяции функции $f$ многочленом $P_n$ степени не выше $n$ на $T$ . Было определено \cite{ciarlet}, что, при достаточно общих ограничениях на множество $T$ из триангуляции и условия интерполяции, имеют место следующие оценки погрешности аппроксимации производных:
$$\left \| D^k f(u) - D^k P_n(u) \right \|_T \le C M_{n+1} \frac {H_{n+1}} {\rho^k},\quad 0 \le k \le n ,$$

где $u \in T \subset R^m$, $P_n(u)$~--- интерполяционный многочлен типа Лагранжа, Эрмита или Биркгофа степени не выше $n$ (степень монома~--- сумма степеней всех его переменных, степень $P_n(u)$~--- максимум степеней его мономов); $H$~--- диаметр $T$; $\rho$~--- радиус шара, вписанного в $T$; $C$~--- константа, не зависящая от $f$ и геометрии $T$. В случае, когда $T$~--- треугольник, правая часть (1) с точностью до абсолютных констант эквивалентна величине
$$C M_{n+1} H^{n+1-k}\sin^{-k}\alpha,$$

где $\alpha$~--- наименьший из углов треугольника. В двумерном случае такая оценка упоминается в \cite{zenisek2016,bramble2016,zlamal2016}.  
Корнеев \cite{korneev2016}, впервые рассматривая высокоточные методы конечных элементов, получил подобную оценку для многомерного случая.

Вместе с тем Женишек \cite{zenisek2016} получил оценку приближения частных производных первого порядка через синус среднего по величине угла. Было показано \cite{babuska2016}, что в некоторых случаях оценки (1) можно улучшить, что позволяет накладывать на триангуляцию более слабые ограничения. Так в \cite{subbotin2016,subbotin22016} показно, что для любого треугольника при $m=2$, $n\ge 2$ и $k=1$ имеются следующие оценки:

$$\left | f(u)-P_n(u) \right | \le C M_{n+1} h^{n+1}, \eqno( 2 )$$
$$\left | Df(u)-DP_n(u)\right | \le C_2 M_{n+1} {\frac {h^{n}} {\sin{\theta}}} , \eqno( 3 )$$
где $u \in T$, $P_n$~--- интерполяционный многочлен типа Лагранжа, $h$~--- наибольшая из сторон $T$, $\theta$~--- наибольший угол, а константы $C$ и $C_2$ не зависят от $f$ и $T$. Указанные оценки позволяют снизить требования к триангуляции, и в случае $R^2$ ограничения налагаются лишь на наибольший угол треугольника.


\section{Построение сетки и реализация алгоритма}

Алгоритм принимает на вход файл изображения, который сохраняется в памяти как \textit{Bitmap} объект с 24 битным качеством цветопередачи в цветовом пространстве $RGB$. Однако работа с изображением в данном виде неудобна, поскольку имеет большую избыточность.%( Рис.~\ref{parrot_fig2}).
%\Figure{0.8\textwidth}{images/parrot2.png}{RGB компоненты изображения\label{parrot_fig2}}
$Y' C_B C_R$~--- семейство цветовых пространств, которые используются для передачи цветных изображений, $Y'$~--- компонента яркости, $C_B$ и $C_R$ являются синей и красной
цветоразностными компонентами. $Y'C_B C_R$ не является абсолютным цветовым
пространством, скорее, это способ кодирования информации сигналов $RGB$.

Сигналы $Y' C_B C_R$ можно получить из соответствующих $RGB$ источников с
помощью следующиx уравнений \cite{miano2016}

$$Y' = 0,299R + 0,587G +0,114B,$$
$$C_B = 128 - 0,168736R-0,331264G+0,5B,$$
$$C_R = 128 + 0,5R - 0,418688G - 0,081312B,$$

после применения которых получаются искомые потоки данных. Компоненты $Y' C_B C_R$ имеют меньшую избыточность в отличие от $RGB$.%(Рис.~\ref{parrot_fig4}).
%\Figure{0.8\textwidth}{images/parrot4.png}{Y’CbCr компоненты изображения\label{parrot_fig4}}

Изображение преобразованное в $Y' C_B C_R$ представляется в виде трех float матриц. Поскольку цветовые компоненты несут гораздо меньше информации, алгоритм
будет работать с яркостной компонентой. Пусть $Y$~--- $(W \! \times \! H)$-матрица значений компонент яркости размерности, где $W$, $H$~--- ширина и высота исходного изображения.

Первым шагом алгоритма генерируется матрица $Y_D$~--- максимумов второй производной для точек из $Y$. Значения из этой матрицы будут использоваться при подсчете оценок погрешности аппроксимации (2), (3).

Рассмотрим $T$~--- один из симплексов триангуляции в $R^2$, $h$~--- наибольшая из сторон, $\theta$~--- наибольший угол. Пусть функция $f$
непрерывна на $T$ вместе со всеми частными производными до второго порядка включительно, причем для любых выбранных задающих направления единичных векторов
$\xi_1,\xi_2$ абсолютные значения производных $D^n_{\xi_1,\xi_2}f$ ограничены на $T$ числом $M$. $P_{n-1}$~--- полином Лагранжа интерполирующий функцию $f$ и ее производные на $T$. Пусть $R(T) = C M h^n$ и $R'(T) = C_2 M \frac {h^{n-1}} {\sin \theta}$~--- оценки погрешности аппроксимации на $T$ (2), (3).

Разобьем T на два меньших треугольника, добавив узлы на середину одной из сторон. На каждом треугольнике считаем $R(T_i) = M^{(i)} h_i^2$ и $R'(T_i) = M^{(i)} \frac {h_i}{\sin \theta_i}$, где $M^{(i)}$~--- максимальное значение $Y_D$ для точек из треугольника.

Поскольку $T$ таким способом можно разделить на три пары меньших треугольников, то требуется выбрать из них наилучшую. Выбор лучшего варианта зависит от реализации. Так можно считать лучшим разбиение $j$, в котором достигается наименьший $R'(T^j_i),\,i = 1,2,\, j=1,\dots,J$. В этом случае $T$ будет делиться так, чтобы достигнуть на одном из треугольников наиболее простое поведение исходного потока.

Выбранная пара треугольников будет новыми ячейками сети. Построение сетки идет путем рекурсивного применения алгоритма к новым ячейкам сети.

\section{Результаты}




Данный алгоритм был опробован на различных несжатых битовых изображениях:
\begin{enumerate}
  \item
    test.bmp~--- изображение с белым фоном, цветные линии, разрешение $2048 \times 1536$;
  \item
    test2.bmp~--- аналогичное test.bmp, разрешение $1024 \times 768$;
  \item
    jellyfish.bmp~--- фотография с изображением медузы, однородный фон, мало мелких деталей, разрешение $1024 \times 768$;
  \item
    desert.bmp~--- фотография с изображением пустыни, много мелких деталей, разрешение $1024 \times 768$;
  \item
    face.bmp~--- фотография лица, разрешение $1600 \times 1200$.
\end{enumerate}


\begin{table}[h]
\begin{center}
\begin{tabular}{|c|c|c|}
\hline
\thd{Изображение} & \thd{Преобразование, сек} & \thd{Создание сетки, сек} \tabularnewline
\hline
test.bmp & 8,5 & 60,7  \tabularnewline
\hline
test2.bmp & 1,6 & 16,5 \tabularnewline
\hline
jellyfish.bmp &  1,6 & 23,5  \tabularnewline
\hline
desert.bmp & 2,0 & 24,6 \tabularnewline
\hline
face.bmp & 4,5 & 44,4 \tabularnewline
\hline
\end{tabular}
\end{center}
\caption{CPU: Intel i5 2400, 1 поток}\label{tab:images2016}
\end{table}

Видно (см. Таблицу \ref{tab:images2016}), что преобразование изображения в основном зависит только от числа точек. В то время как создание сетки зависит и от числа деталей, и от размера изображения. 

Поскольку на каждом шаге алгоритма разделение одной ячейки сетки не зависит от других ячеек, то алгоритм построения сетки легко распараллелить. Также, поскольку при преобразовании изображения каждая точка обрабатывается независимо друг от друга, то преобразование для каждой точки можно считать в отдельном потоке. Аналогично можно распараллелить расчет матрицы с максимальными производными.

Полученные в результате работы сетки можно использовать как для последующей интерполяции и восстановления изображения (в случае, если алгоритм выполнялся пока $R'(T)$ не стал меньше определенного значения), так и для последующего укрупнения сетки и выделения основного и вейвлетного потоков. 

\section{Заключение}

Предложен новый алгоритм, позволяющий построить неравномерную сетку на двумерном потоке данных, которая приспосабливается к изменениям в потоке. Разработано приложение, позволяющее строить сетку на любом несжатом битовом
изображении и позволяющее вывести полученную сетку в наглядном виде.
Также проверена возможность восстановления изображения при помощи интерполяции по данной сетке.
Предложен вариант параллельной реализации алгоритма.

В дальнейшем планируется использовать сетку, построенную данным алгоритмом, для вейвлетного разложения
двумерного потока данных.

\renewcommand\refname{Литература}
\begin{thebibliography}{10}
\bibitem{demyan2016} Демьянович~Ю.~К., Зимин~А.~В. Аппроксимации курантова типа и их вэйвлетные разложения // Проблемы математического анализа. 2008. Вып.~37. С.~3–-22.
\bibitem{ciarlet2016} Ciarlet~P.~G., Raviart~P.~A. General Lagrange and Hermite interpolation in $R^n$ with applications to finite element methods // Arch. Rat. Mech. Anal. 1972. Vol. 46, No~3. P. 177--199.
\bibitem{zenisek2016} Zenisek~A. Interpolation polynomials on the triangle // Numer. Math. 1970. Vol. 15. P. 283--296.
\bibitem{bramble2016} Bramble~J.~H., Zlamal~M. Triangular elements in the finite element method // Math. Comp. 1970. Vol. 24, No~112. P. 809--820.
\bibitem{zlamal2016} Zlamal~M., Zenisek~A. Mathematical aspect of the finite element method // Technical, physical and mathematical principles of the finite element method / eds. V. Kolar et al. Praha: Acad. VED. 1971. P. 15--39.
\bibitem{korneev2016} Корнеев~В.~Г. Схемы метода конечных элементов высоких порядков точности. Л.: Изд-тво Ленинградского университета, 1977. 206~с.
\bibitem{babuska2016} Babuska~I., Aziz~A.~K. On the angle condition in the finite element method // SIAM J. Numer. Anal. 1976. Vol 13, No~2. P. 214--226.
\bibitem{subbotin2016} Субботин~Ю.~Н. Многомерная кусочно-полиномиальная интерполяция // Методы аппроксимации
и интерполяции / под ред. А.~Ю.~Кузнецова. Новосибирск: ВЦ СО АН, 1981. C. 148--153.
\bibitem{subbotin22016} Субботин~Ю.~Н. Зависимость оценок многомерной кусочно-полиномиальной аппроксимации от
геометрических характеристик триангуляции // Тр. МИАН. 1989. Т. 189. С. 117--137.
\bibitem{miano2016} Миано~Дж. Форматы и алгоритмы сжатия изображений в действии. М.: Триумф, 2003. 336 с. 
\end{thebibliography}

\end{document}
