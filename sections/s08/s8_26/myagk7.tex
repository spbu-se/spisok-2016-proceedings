\documentclass{spisok-article}

%\RequirePackage[unicode]{hyperref}
\RequirePackage{authblk}
\RequirePackage{booktabs}
\RequirePackage{indentfirst}
\RequirePackage{titlesec}
\RequirePackage{graphicx}
\RequirePackage[table,xcdraw]{xcolor}
\RequirePackage{listings}
\RequirePackage{makecell}


\title{Минимальные квадратичные сплайны и интервальные оценки}

\author{Бурова И.Г., д.ф-м.н., профессор СПбГУ burovaig@mail.ru\newline{}Вартанова А.А., аспирантка СПбГУ 104smail@mail.ru\newline{}Мягких И.П, rilyam@mail.ru}

% \usepackage{amsfonts,amsmath,amsthm}

\begin{document}

\maketitle

\begin{abstract}
%
В данной статье рассматривается интервальная оценка приближения
 минимальными кусочно-квадратичными  сплайнами лагранжевого типа  и представлен
  интервальный алгоритм  нахождения корня функции методом деления отрезка пополам, с использованием аппроксимации функции кусочно-квадратичным сплайном
\end{abstract}

\section{Введение}

Инструменты интервальной арифметики используются при решении различных задач [1]. В данной работе получена интервальная оценка  приближения функции, построенного с помощью квадратичных полиномиальных сплайнов [2-3]. Здесь будет рассмотрена интервальная интерпретация поиска корней аппроксимирующей функции методом деления отрезка пополам.

\section{Интервальная аппроксимация правыми кусочно-квадратичными сплайнами}

Пусть $a$, $b$ --- вещественные числа, $n$ --- натуральное число. Построим сетку упорядоченных узлов
$\{ x_j \}$, $j=0,\ldots,n$   на промежутке $[a,b]$.

Пусть функция $u(x)$, $u\in C^3[a,b]$ дана значениями в узлах сетки.
В этом случае можно ее  аппроксимировать
 правыми минимальными сплайнами второй степени.

 Приближения правыми квадратичными сплайнами на
промежутке $[x_j, x_{j+1}]$,  $j=0$, $\ldots$, $n-2$, имеют вид

$$\widetilde{u}(x)=ABu(x_{j-1}) -ACu(x_{j})+BCu(x_{j+1}), $$
где $$A=\frac{x-x_{j+1}}{  x_{j-1}-x_{j}} , \ \
B=\frac{x-x_{j} }{ x_{j-1}-x_{j+1}}, \ \
 C=\frac{x-x_{j-1}}{  x_{j}-x_{j+1}}.$$

%\bigskip
%\vskip0.5cm


 Рассмотрим интервальный аналог этой аппроксимации. Заменим  $x$ на интервал
  $X=[x_j, x_{j+1}]$,  тогда приближение правыми квадратичными
  сплайнами на промежутке $[x_j, x_{j+1}]$   нетрудно привести к виду:
		$$\widetilde{u}(X)=ABu(x_{j-1}) -ACuu(x_{j})+BCu(x_{j+1}), $$
где $$A=\frac{X-x_{j+1}}{  x_{j-1}-x_{j}} , \ \
B=\frac{X-x_{j} }{ x_{j-1}-x_{j+1}}, \ \
 C=\frac{X-x_{j-1}}{  x_{j}-x_{j+1}}.$$

Преобразуем последнее выражение к виду:
$$\widetilde{u}(X)=d_2X^2+d_1X+d_0,$$ где
  $$d_2= \frac{u_{j-1}}{(x_{j-1}-x_j)(x_{j-1}-x_{j+1})}-
  \frac{u_{j}}{(x_{j-1}-x_j)(x_j-x_{j-1})}+$$ $$+\frac{u_{j+1}}
  {(x_{j-1}-x_{j+1})(x_j-x_{j-1})},$$
$$d_1=  -\frac{A}{(x_j-x_{j-1})^2(-x_{j-1}+x_{j+1})},$$
где $A=(u_{j-1}x_{j+1}(x_j-x_{j-1})+u_{j-1}x_j(x_j-x_{j-1})
+u_j(x_{j+1}^2-x_{j-1}^2)-u_{j+1}(x_j^2-x_{j-1}^2))$,
$$d_0= \frac{x_{j+1}x_ju_{j-1}}{(x_{j-1}-x_j)(x_{j-1}-x_{j+1})}
-\frac{x_{j+1}x_{j-1}u_j}{(x_{j-1}-x_j)(x_j-x_{j-1})} +$$$$
+\frac{x_jx_{j-1}u_{j+1}}{(x_{j-1}-x_{j+1})(x_j-x_{j-1})}.$$

\section{Интервальные оценки}
На практике  $\widetilde{u}(X)$  можно представить различными способами:

1) $\widetilde{u}(X)=d_2X^2+d_1X+d_0,$

        2) $\widetilde{u}(X)=d_2XX+d_1X+d_0,$

3)   $\widetilde{u}(X)=(d_2X+d_1)X+d_0,$ (по схеме Горнера.)


\noindent Нетрудно показать, что справедлива теорема.

\bigskip
{\it Теорема.}
Пусть  $X, Z$ --- интервалы. Для интервальных оценок $\widetilde{u}(X)$

 1) $Z_1=d_2X^2+d_1X+d_0,$

        2) $Z_2=d_2XX+d_1X+d_0,$

3)   $Z_3=(d_2X+d_1)X+d_0,$

 \noindent справедливы соотношения

1)  $Z_1\subseteq Z_2$ ,

2)  $Z_3\subseteq Z_1$, в случае, если интервал $X$  не содержит нуля.

3) Если  $0\in X$ , то $Z_2\subseteq Z_3$.

Доказательство очевидно.

\bigskip
{\it Замечания.}

 1. Использование унарной операции возведения в квадрат при построении интервальной оценки аппроксимации кусочно-квадратичными сплайнами предпочтительнее, чем бинарная операция умножения.

2. Если интервал  $X$ не содержит нуля, то схема Горнера предпочтительнее.


\bigskip
{\it Пример 1.}

Рассмотрим функцию $u(x)=x^5$  и интервал $X=[0.1, 0.2]$,
$d_2 = 0.015,	d_1 = -0.0014, 	d_0 = 0.0$.
В случае интервальной оценки  $\widetilde{u}(X)=d_2X^2+d_1X+d_0$
получаем $\widetilde{u}(X)=[-0.00013,0.00046]$.

В случае интервальной оценки  $\widetilde{u}(X)=d_2XX+d_1X+d_0$ получаем
$\widetilde{u}(X)=[-0.00013,0.00046]$.


В случае интервальной оценки  $\widetilde{u}(X)=(d_2X+d_1)X+d_0,$ получаем
$\widetilde{u}(X)=[0.00001,0.00032]$.

Видно, что только в случае использования схемы Горнера
получается интервал совпадающий с точным значением.

\bigskip
{\it Пример 2.} Рассмотрим эту же функцию $u(x)=x^5$ и интервал $X=[-0.1, 0.1]$.



$d_2 = -0.03,	d_1 = 0.0001, 	d_0 = 0.0003$.
В случае интервальной оценки  $\widetilde{u}(X)=d_2X^2+d_1X+d_0$ получаем
$\widetilde{u}(X)=[-0.00001, 0.00031]$.

В случае интервальной оценки  $\widetilde{u}(X)=d_2XX+d_1X+d_0$ получаем
$\widetilde{u}(X)=[0.00001,0.00061]$.


В случае аппроксимации получаем  $\widetilde{u}(X)=(d_2X+d_1)X+d_0,$ получаем
$\widetilde{u}(X)=[-0.00001,0.00061]$,


В данном примере схема Горнера даёт такой же результат как использование бинарной операции.
При этом использование унарной операции возведения в квадрат даёт более точную аппроксимацию.


\section{Метод деления отрезка пополам}

Интервальный аналог  алгоритма метода деления отрезка пополам состоит в следующем. На входе алгоритма --- интервал  $X^{(0)}$.
Пусть функция $U$ непрерывна на интервале $X^{(0)}=[x_1^{(0)}, x_2^{(0)}]$
  и значения $U(X)$ на концах интервала противоположны по знаку.
Предположим, что интервал $X^{(0)}$ содержит ровно один корень функции $U$.

 Алгоритм нахождения корня:

1) Разделим интервал $(X^{(0))}$ на два интервала:
	 $X_1^{(1)}=[x_1^{(0)}, (x_1^{(0)}+x_2^{(0)})/2]$ и $X_2^{(1)}=[(x_1^{(0)}+x_2^{(0)})/2, x_2^{(0)}]$

2) В случае, если  $U(X_1^{(1)})\cap [-\varepsilon,\varepsilon]$ не пусто
 (где $\varepsilon>0$  , то корень функции $U$
  лежит в интервале $X_1^{(1)}$. Тогда положим $X^{(1)}=X_1^{(1)}$.

Если  $U(X_2^{(1)})\cap [-\varepsilon,\varepsilon]$ не пусто, то корень функции $U$
лежит в интервале $X_2^{(1)}$. Тогда заменим  положим $X^{(1)}=X_2^{(1)}$.

Произведём такие же действия с полученным интервалом $X^{(1)}=[x_1^{(1)}, x_2^{(1)}]$.
Ширина интервала   $d( X^{(0)})$ сократилась вдвое.	

Интервальная интерпретация данного алгоритма обладает всеми преимуществами интервальной оценки погрешностей.

Процесс деления интервала продолжается до тех пор, пока ширина не станет достаточно мала.
  Значение искомого корня принадлежит интервалу, полученному на выходе алгоритма.

  \bigskip
    {\it Пример 3}.
Рассмотрим полином $u(x)=x^7+3x^6-4x^5-12x^4-x^3-3x^2+4x+12$.
 на интервале $[1.8, 2.4]$.
Предположим, что функция $u$ известна в точках 1.2, 1.8, 2.4, т.е. нам заданы
значения $u(1.2)=-11.5433472$, $u(1.8)=-34.6472448$, $u(2.4)= 305.8159104$.
Необходимо найти корень $u$ как можно более точно.
Строим квадратичный сплайн на промежутке $X=[1.8, 2.4]$.
Выбираем $X_1$ как можно более узкий и содержащий корень квадратичного сплайна.
Пусть $X_1=[1.86,2.04]$.
Построим интервальную оценку $\tilde{U}(X_0)$, $X_0=[1.86,2.04]$, по схеме Горнера, получаем
$\tilde{U}(X_0)=[-16.96145, 57.9099709440005]$.

Находим корень $\tilde{U}(X_0)$ методом интервальной модификации деления отрезка пополам.
На первой итерации получаем, что содержится в $X_1=[ 1.8,2.10]$. Ширина последнего интервала равна
$d=0.0056$.

На четвертой итерации получаем, что корень квадратичного сплайна $\tilde{X}$, найденный интервальной корень  модификацией, находится в промежутке $X_4=[ 1.9050, 1.9106]$.
 Нетрудно видеть что этот
промежуток действительно содержит корень $\tilde{u}$, который равен 1.90852.
Заметим,что корень исходной функции $u$ на [1.8, 2.4] равен 2.

Отметим, что оценка погрешности приближения квадратичным сплайном составляет
$|{u}(x)-\tilde{u}(x)|_{[1.2,2.4]}\le 98651 \ h^3/3!$. При выбранном $h=0.6$ эта оценка
оказывается очень большой - 355. 
Если можно выбрать $X$ более узким, ширины $2h$, $h\le 0.01$, и содержащим корень $u$,
 (при этом предполагаются
известными значения функции в требуемых точках), то можно 
надеяться получить интервал содержащий корень исходной функции ширины, обеспечивающей по крайней мере две верные цифры в мантиссе величины корня
исходной функции $u$.

\section{Заключение}

В статье были получены интервальные оценки для аппроксимации функций минимальными полиномиальными сплайнами второй степени, составлен алгоритм для интервального аналога метода деления отрезка пополам с применением аппроксимации сплайнами второй степени.

\makeatletter\renewcommand{\refname}{\intl@references}\makeatother
\begin{thebibliography}{8}

 \bibitem{1}	Г. Алефельд, Ю. Херцбергер. Введение в интервальные вычисления,
 Москва: Изд-во Мир, - 1987.

\bibitem{2}	И. Г. Бурова. Минимальные вещественные и комплексные сплайны", СПб, - 2013.

\bibitem{3} И.Г. Бурова, Ю.К.Демьянович.  Минимальные сплайны и их приложения.
 - СПб: Изд-во С.-Петерб. университета, - 2010. - 364 с.
\end{thebibliography}
\end{document} 
