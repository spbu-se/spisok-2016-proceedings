\documentclass{spisok-article}

\newcommand*{\No}{\textnumero}

\title{Применение рандомизированных алгоритмов для достижения дифференцированного консенсуса в мультиагентных сетях\thanks{Работа выполнена при поддержке РФФИ, проект №16-07-00890.}
}

\author{
  Иванский Ю. В.,
  аспирант кафедры системного программирования СПбГУ,
  ivanskiy.yuriy@gmail.com
}

\begin{document}

\maketitle

\begin{abstract}
В статье рассматривается задача достижения дифференцированного консенсуса в мультиагентной сети, обрабатывающей разных классов. Исследуется проблема достижения консенсуса  для каждого класса по отдельности. Предполагается, что сеть работает в условиях переменной структуры связей, помех и задержек при передаче информации по каналам связи. В работе обсуждается рандомизированная управляющая стратегия  перераспределения заданий, позволяющая достичь дифференцированного консенсуса при указанных условиях. Для оптимизации выбора величины размера шага алгоритма (коэффициента усиления) предлагается использовать рандомизированный алгоритм типа SPSA. 
\end{abstract}

\section{Введение}

%На практике во время работы системы одни задания могут иметь более высокую важность, чем другие. Это может быть вызвано необходимостью передачи служебной информации для поддержания работы системы; некоторые задания могут иметь более высокую срочность, в то время как для других заданий время их выполнения в системе не так критично; у системы могут быть несколько групп пользователей, которым требуется разный уровень обслуживания. В таких ситуациях для различения заданий разных классов вводят приоритеты, а для принятия решения о выборе задания на исполнение учитывается уровень его приоритета. 

%В последнее время наблюдается все возрастающее применение беспилотных летательных аппаратов, как в военных, так и в гражданских целях, проводятся многочисленные эксперименты по организации взаимодействия групп мобильных роботов, набирает популярность концепция интернета вещей~--- соединения различных объектов в сеть для организации их взаимодействия. Перечисленные тенденции актуализируют разработку и применение мультиагентного подхода в управлении группой объектов, поскольку, с одной стороны, подобными группами зачастую неэффективно управлять централизованно, с другой~--- объекты в таких группах обладают определенной степенью интеллектуальности, то есть способны самостоятельно организовывать и поддерживать взаимодействие внутри группы.

В ходе выполнения диссертационной работы <<Рандомизированные алгоритмы в задачах мультиагентного взаимодействия>> выполнялись разработка и исследование алгоритмов повышения эффективности децентрализованного перераспределения между узлами вычислительной сети очередей задач с разными уровнями приоритетов в условиях переменной структуры связей, помех и задержек при передаче информации. 
Основные результаты исследований отражены в статьях~\cite{vest15,kio15,msc14,ecc15,micnon15,codit16,soi9-1,soi9-2,vspu14,ts14,soi10,soi11}

\section{Постановка задачи}
Пусть система образована $n$ агентами, взаимодействующими друг с другом, и множеством заданий $m$ различных классов, которые должны быть выполнены системой. Задания поступают в систему на разных агентов в различные дискретные моменты времени $t = 1, \ldots$. % Агенты выполняют приходящие задания параллельно. Задания могут быть перераспределены среди агентов за счет использования обратной связи. %Заметим, что выполнение задания не может быть прервано после того, как оно было назначено агенту.

%Не умаляя общности,
Сопоставим каждому агенту номер $i,\; i= 1, \ldots, n,$. Пусть $N = \{ 1, \ldots, n\}$ --- множество всех агентов в системе. Топология сети может изменяться со временем и моделируется последовательностью ориентированных графов $\{(N, E_{t})\} _{t\ge 0} $, где $E_t %\subset E
$
%обозначает
---
множество ребер в графе $(N, E_{t})$ в момент времени~$t$. Соответствующие матрицы смежности обозначим $A_t = [a^{i, j}_t]$. %, где $a^{i, j}_t > 0$ если агент $j$ соединен с агентом $i$ и $a^{i, j}_t = 0$ в противном случае. %Здесь и далее, верхний индекс обозначает номер соответствующего агента (а не возведение в степень). 
%Матрица  $A_t =[a^{i, j}_t]$ является матрицей смежности графа сети в момент времени~$t$. Обозначим этот граф ${\cal G}_{A_t}$.
\if 0
Будем использовать
%Для введения некоторых свойств сетевой топологии потребуются
следующие
%обозначения
определения
из теории графов. %Положим по определению
\textit{Взвешенная полустепень захода} узла $i$ равна сумме $i$-й строки матрицы $A$: $d^i(A)=\sum_{j=1}^n a^{i, j}$; $D(A) ={\rm diag}\{d^i(A)\}$~--- соответствующая диагональная матрица; $d_{\max}(A)$~--- максимальная полустепень захода в графе ${\cal G}_A$
%. Обозначим
${\cal L}(A) = D(A) - A$ --- \textit{лапласиан} графа ${\cal G}_A$; $\cdot^{\rm T}$~--- операция векторного или матричного транспонирования; $||A||$~--- Евклидова норма: $||A||= \sqrt{\sum_i \sum_j (a^{i, j})^2}$; $Re(\lambda_2(A))$~--- действительная часть второго по величине собственного числа матрицы  $A$; $\lambda_{\max}(A)$~--- это наибольшее собственное число матрицы $A$.
%Будем говорить, что
Орграф ${\cal G}_B$ является подграфом ${\cal G}_A$, если $b^{i, j} \leq a^{i, j}$ для всех $i,j \in N$.
Говорят, что граф ${\cal G}_A$ содержит \textit{остовное дерево}, если существует направленное дерево ${\cal G}_{tr}=(N,E_{tr})$, являющееся подграфом ${\cal G}_A$.
\fi
Предположим, что задачи (задания) относятся к различным классам (приоритетам) $k = 1, \ldots, m$ и у каждого агента есть $m$ очередей~--- по одной на задания каждого класса.

Поведение агента $i \in N$ задают две характеристики:
%\begin{itemize}
%\item 
1) вектор размерности $m$, состоящий из длин очередей заданий $\mathbf q_{t}^{i} = [q^{i, k}_t] $ в момент времени~$t$,  $k$-й элемент которого равен числу заданий $k$-го класса $k=1, \ldots, m$;
%\item 
2) производительность $p^{i}$.
%\end{itemize}

%Агенты, имеющие каждый свою производительность (или число операций, выполняемое агентом за единицу времени) должны распределить ее среди всех классов заданий таким образом, чтобы, с одной стороны, обеспечить очередность выполнения заданий согласно их приоритетам, a с другой стороны (принимая во внимание ``проблему голодания'') чтобы задания с низким приоритетом  не простаивали ``бесконечно'', дожидаясь своей очереди на исполнение.
%Агенты могут иметь разную производительность, то есть число операций, выполняемое агентом за единицу времени, может различаться. 
Агенты должны распределить количество вычислительных операций между заданиями различных классов таким образом, чтобы, с одной стороны, обеспечить очередность выполнения заданий согласно их приоритетам, a с другой стороны, чтобы задания с низким приоритетом  не простаивали ``бесконечно'', дожидаясь своей очереди на исполнение.
%Такого эффекта можно добиться, введя вероятностные приоритеты. 
Каждому классу заданий поставим в соответствие долю производительности $P_k, \; k = 1, \ldots, m $, одинаковую для конкретного класса $k$ для всех агентов. На каждом агенте задания из очередей будем выбирать случайно с вероятностью $
\tilde p_t^{i, k} = \frac{P_k}{\sum_{q_t^{i, l} > 0} P_l}$, если $q_t^{i, k}>0$, а иначе $ \tilde p_t^{i, k} = 0$. 

Здесь $\tilde p_t^{i, k}$~--- это вероятность выбора на исполнение агентом~$i$ задания класса~$k$ в момент времени~$t$. Таким образом, чем больше $P_k$, тем выше вероятность выбрать на исполнение задание класса $k$. Отсюда, производительность агента распределяется между всеми классами заданий следующим образом:
%\begin{equation}
$
p_{t, av}^{i, k} = \tilde p_t^{i, k}  p^i.
$
%\end{equation}
Здесь $p_{t, av}^{i, k}$ обозначает число операций, выделяемое \textit{в среднем} на агенте~$i$ для выполнения заданий класса~$k$ при текущей вероятности выбора их на исполнение в момент времени~$t$. Обозначим $p_t^{i, k}$ количество операций, выделенное агентом~$i$ на задания класса~$k$ в момент времени~$t$. %Заметим, что по определению $\tilde p_t^{i, k}$, если в момент времени~$t'$ очередь из заданий класса~$k'$ на агенте~$i'$ оказалась пуста, то $p_{t'}^{i', k'}$ операций будут распределены между другими классами заданий в отношении, равном отношению их долей производительности $P_k, \; k \ne k'.$

Для всех $i \in N,\; t = 0,  1, \ldots$, динамика сетевой системы имеет вид: %следующим образом:
%\begin{equation}
%\label{dyn}
$
\mathbf q_{t+1}^{i} = \mathbf q_{t}^{i} - \mathbf p_t^{i} + \mathbf z_{t}^{i} + \mathbf u_{t}^{i},
$
%\end{equation}
где $\mathbf p_t^i = [p_t^{i, k}]$, и
$\mathbf z_{t}^{i} = [z_t^{i, k}]$~---
векторы размерности $m$, $k$-й элемент $z_t^{i, k}$ обозначает число новых заданий класса $k$, поступивших в систему на агента $i$ в момент времени $t$; $\mathbf u_{t}^{i} \in {\mathrm R}^m$ является вектором управляющих воздействий размерности $m$. %(состоит из перераспределенных на агенте $i$ заданий класса $k$ в момент времени $t$), который следует выбирать основываясь на информации о длинах очередей на соседних агентах $\mathbf q_{t}^{j},\; j \in N^i_t$, где $N^i_t$~--- это множество $\{j \in N: a^{i, j}_t > 0\}$.

Для топологии $\{N^i_t, i \in N\}$ определим стоимость $C(\{N^i_t, i \in N\}) = \max_{i \in N} \sum_{j \in N_t^i} a_t^{i, j}.$

%Будем рассматривать протоколы управления, удовлетворяющие определенному ограничению на стоимость топологии для каждого отдельного уровня приоритета. Положим $p^{i} \neq 0, \; \forall i \in N$ и $P_k \neq 0, \; k = 1, \ldots, m$.

\textit{Цель управления состоит в том, чтобы поддерживать сбалансированную (равную) загрузку по всей сети для всех уровней приоритета, удовлетворяя при этом требование на ограничение стоимости.}

В такой постановке можно решать задачу достижения консенсуса для состояний агентов $\mathbf x_t^i= [x_t^{i, k}]$, где $x_t^{i, k} = q_t^{i, k} / p_{t, av}^{i, k}.$
\if 0
$$
x_t^{i, k} = \left\{
                \begin{array}{ll}
                  q_t^{i, k} / p_{t, av}^{i, k}, & \hbox{если  } {\tilde p}_t^{i, k}>0; \\
                  0, & \hbox{иначе.}
                \end{array}
              \right.
$$
\fi
%\emph{Подчеркнем, что $\mathbf x_t^i$ является вектором состояния, образованным из состояний для каждого из $m$ классов.}

%Для балансировки загрузки сети (чтобы повысить общую производительность сети и уменьшить таким образом время завершения выполнения всех заданий) естественно использовать протокол перераспределения заданий во время работы сети.

Будем считать, что для формирования управляющей стратегии $\mathbf u_t^i$ каждый агент $i \in N$ опирается на зашумленные данные о состояниях соседей, которые также могут приходить с задержкой: $\mathbf y_{t}^{i, j} = \mathbf x_{t - s_{t}^{i, j} }^{j}  + \mathbf w_{t}^{i, j},\; j \in N_{t}^{i},$
где $\mathbf w_{t}^{i, j}$~---  вектор помех, $0 \le s_{t}^{i, j} \le \bar s$~--- целочисленные задержки, а $\bar s$~---  максимально возможная задержка.

\subsection{Стоимостные ограничения на топологию и рандомизированная декомпозиция топологии}

%Пусть $({\Omega},{\cal F},{P})$~--- вероятностное пространство, образованное пространством элементарных событий, набором всех возможных событий, и вероятностной мерой соответственно, $\mathrm{E}$~--- символ математического ожидания.

%Предположим, что графы ${\cal G}_{A_t}$, $t = 1, \ldots$ независимы и одинаково распределены, то есть случайные события появления ребра $(j, i)$ независимы и случайно распределены для определенного $(j, i)$. Обозначим  $a_{av}^{i, j}$ --- средние значения (математические ожидания) $a_{t}^{i, j}$,  $A_{av}$ --- соответствующую матрицу смежности.
%Предположим, что граф ${\cal G}_{A_{av}}$ содержит остовное дерево.

Задания обладают различными приоритетами и для каждого приоритета определена максимальная разрешенная стоимость сетевого графа. В каждый момент времени $t$ будем рассматривать $m$ способов (которые могут различаться и каждый из которых соответствует одному уровню приоритета) выбрать подграф ${\cal G}_t^k: \; {\cal G}_t^m \subset {\cal G}_t^{m-1} \subset \ldots \subset {\cal G}_t^1$ графа ${\cal G}_{A_t}$, позволяющий использовать протокол для перераспределения заданий приоритета $k, \; k = 1, \ldots, m$. Обозначим $B^k_t$ соответствующие матрицы смежности. %Заметим, что один из возможных способов выбора ${\cal G}_t^k$~--- использовать ${\cal G}_{A_t}$ для всех $k$.

Пусть $c_k, \; k=1, \ldots, m$,
%обозначает
---
максимальная средняя стоимость сетевых связей для заданий с приоритетом $k$. Положим, $c_1 \geq c_2 \geq \ldots c_m >0$.

{\bf Определение} Будем говорить, что декомпозиция топологии сети $\{{\cal G}_t^k\}$ удовлетворяет ограничениям на среднюю стоимость $\{c_k\}$, если для каждого класса приоритета $k$ выполнено: $
d_{\max}(B_{av}^k) = \mathrm{E} d_{\max}(B_t^k)  = \mathrm{E} \max_{i \in N} \sum_{j \in N_t^{i, k}} b_t^{i, j, k} \leq c_k,$
где $N_t^{i, k}$~--- множество соседей агента $i$ в момент времени $t$, образованное в соответствии с топологией ${\cal G}_t^k$.

%Будем рассматривать протоколы управления, удовлетворяющие определенному ограничению на стоимость топологии для каждого отдельного уровня приоритета. 

\section{Протокол управления}

Рассмотрим следующее семейство протоколов. Для каждого $k = 1, \ldots, m$ для декомпозиции топологии $\{{\cal G}_t^k\}$ для стоимостных ограничений $\{c_k\},\;c_k>0$ определим
\begin{equation}
\label{Nat_7}
{u}_{t}^{i, k} = \gamma p^{i, k}_{t, av} \sum_{j \in  \bar N_{t}^{i} } b_{t}^{i, j} (y_{t}^{i, j, k} -  x_{t}^{i, k}),
\end{equation}
где $\gamma >0$~--- это шаг протокола управления, а $\bar N_{t}^{i} \subset N_{t}^{i}$~---
множество соседей узла $i$ %(заметим, что можно использовать не все доступные связи, а лишь некоторое их подмножество), 
$b_t^{i, j}$~--- коэффициенты протокола. Используя протокол (\ref{Nat_7}), система работает таким образом, что задания одного приоритета распределяются между агентами равномерно. %Пусть $B_t = [b_t^{i, j}]$~--- матрица протокола перераспределения заданий в момент времени~$t$. (Положим $b_t^{i, j}=0$, когда $a_t^{i, j} = 0$ или $j \notin \bar N_{t}^{i} $.)
%По построению матрицы $B_t$, соответствующий граф ${\cal G}_{B_{t}}$ большую часть времени будет иметь такую же топологию, как граф ${\cal G}_{A_t}$, задаваемый матрицей $A_t$, или более разреженную.

Динамика системы с обратными связями, функционирующей по протоколу~(\ref{Nat_7}), будет иметь следующий вид:
\begin{equation}
\label{Nat_12}
\mathbf x_{t+1}^{i} = \mathbf x_{t}^{i} - \tilde{\mathbf r}_{t}^{i} + \tilde{\mathbf z}_{t}^{i} + \gamma \sum_{j \in \bar N_{t}^{i} } b_{t}^{i, j} (\mathbf y_{t}^{i, j} - \mathbf x_{t}^{i}),
\end{equation}
где вектор $\tilde{\mathbf r}_{t}^{i} = [{\tilde r}_t^{i, k}]$, $ {\tilde r}_t^{i, k} = p_t^{i, k}/\tilde p_t^{i, k} $ и вектор $\tilde{\mathbf z}_{t}^{i} = [{\tilde z}_t^{i, k}]$, $ {\tilde z}_t^{i, k} = z_t^{i, k}/{\tilde p_t^{i, k}}.$

\section{Основной результат}

Пусть
%задано вероятностное пространство $({\Omega},{\cal F},{P})$ на пространстве элементарных событий, множестве всех событий и вероятностной мере соответственно, а $\mathrm{E}$~--- символ математического ожидания.
%Предположим, что графы ${\cal G}_{B_t}$, $t = 1, \ldots$ случайные независимые и одинаково распределенные, то есть случайные события появления ребра $(j, i)$ независимы и одинаково распределены для фиксированного ребра $(j, i)$. Обозначим за $b_{av}^{i, j} $ средние арифметические (математические ожидания) $b_{t}^{i, j} $, а за $B_{av}$~--- соответствующую матрицу смежности.
%Предположим, что
выполнены следующие условия.
%\begin{itemize}
%\item
\newline $\bullet$
{\bf A1}. Граф ${\cal G}_{B_{av}}$ является сильно связным.%имеет остовное дерево.
%\item
\newline $\bullet$
{\bf A2}. {\bf a)} Для любых $i \in N,\; j \in N_t^{i} $ векторы помех наблюдений $\mathbf w_{t}^{i, j}$ центрированные, независимые и одинаково распределенные случайные векторы с ограниченной дисперсией: $\mathrm{E} (\mathbf w_{t}^{i, j})^2\leq \mathbf \sigma_w^2$.
\newline
{\bf b)} Для любых $i \in N, j \in N^{i}_{\max}=\cup_t \bar N^{i}_{t}$ появление ``изменяющегося во времени'' ребра $(j, i)$ в графе ${\cal G}_{B_t}$~--- независимое случайное событие. Для всех  $i \in N,\; j \in N_t^i$ веса $ b_{t}^{i, j}$ в протоколе управления~--- независимые случайные величины с математическим ожиданием: $\mathrm{E} b_{t}^{i, j} = b^{i, j}$ и ограниченной дисперсией: $\mathrm{E} (b_{t}^{i, j} - b^{i, j})^2 \leq \sigma_{b}^2$.
Обозначим $B_{av}=\mathrm{E} B_{t}$~--- соответствующую усредненную матрицу смежности.
\newline
{\bf c)} Для любых $i \in N, j \in N^{i}$ существует конечная целая неотрицательная величина $\bar s \in \mathrm Z^+$: $s_{t}^{i, j} \leq \bar s$ с вероятностью $1$, и целочисленные задержки $s_{t}^{i, j} $ являются независимыми одинаково распределенными случайными величинами, принимающими значения $l=0, \ldots, \bar s$ с вероятностью $S_l^{i,j}$.
\newline
{\bf d)} Для любых  $k=1, \ldots, m,\; i \in N,\;t=0, 1, \ldots$ случайные величины $z_{t}^{i, k}$ независимы и имеют матожидания: $\mathrm{E} z_{t}^{i, k} = {\bar {z}}^k$, не зависящие от $i$, и дисперсии: $\mathrm{E} (z_{t}^{i, k} - {\bar {z}}^k)^2 \leq \sigma_{z,k}^2$.
\newline
{\bf e)} Для любых $ i \in N,\;t=0, 1, \ldots$ случайные векторы ${\mathbf {p}}_t^{i}$ независимы и состоят  из независимых случайных компонент. Случайные величины $\tilde r_{t}^{i, k}, \;k=1, \ldots, m,$ имеют математические ожидания: $\mathrm{E} \tilde r_{t}^{i, k} = {\bar {r}}^k$, не зависящие от~$i$.

Кроме того, все упомянутые в предположениях {\bf A2.a--A2.e} независимые случайные величины и векторы не зависят друг от друга.
\if 0
Заметим, если предположения {\bf A2.b} и {\bf A2.c} выполняются, то усредненная матрица ${\bar B}_{av}=\mathrm{E} {\bar B}_t$, состоит из элементов
\begin{equation}
{\bar b}_{av}^{i, j, k} =
\left\{
	\begin{array}{l}
		S^{i, j \bmod n}_{j \div n} b^{i, j \bmod n},  \hbox{ если } i\in N, \; j \bmod n \neq 0 \\
		S^{i, n}_{j \div n} b^{i, n},  \hbox{ если } i\in N, \;  j \bmod n = 0 \\
		1/\gamma_k,  \hbox{ если } i = n+1,\ldots, \bar n, \; j= i-n, \\
		0,  \hbox{ иначе.}
	\end{array}
\right.
 \end{equation}
Здесь $\bmod$~--- операция взятия остатка от деления, а $\div$~--- деление без остатка.
Если $\bar s = 0$, то ${\bar B}_{av}={ B}_{av}$.
\fi
\newline
$\bullet$
%\item
{\bf A3.} Размер шага протокола управления $\gamma>0$ удовлетворяет следующим условиям: $$\delta = |Re(\lambda_2({\cal L}(\bar B_{av}\otimes I_m)))| - \gamma Re(\lambda_{\max}(Q)) >0, \; \;
0 < \gamma < \frac{1}{\max\{d_{\max}(B_{av}),  \delta\}}$$
где
$Q = \mathrm{E} C_t^{\rm T}C_t,\;\; C_t ={\cal L}(\bar B_{av}\otimes I_m)-{\cal L}({\bar B}_t\otimes I_m).$

%\end{itemize}

\noindent{\it Усредненная модель.}
Пусть $\mathbf x_0^{\star}$~--- средневзвешенный вектор начальных состояний размерности $m$:
$
\mathbf x_0^{\star} = \frac{\sum_i g_i \mathbf x_0^{i}}{\sum_i g_i},
$
где $g^T$~---  левый собственный вектор матрицы $B_{av}$,
и $\{\mathbf x_t^{\star}\}$~--- траектории усредненных систем
\begin{equation}\label{aver_syst}
 \mathbf x_{t+1}^{\star} = \mathbf x_t^{\star} + {\bar {\mathbf{z}}} - {\bar {\mathbf {r}}},\;{\bar {\mathbf{z}}}=[{\bar {z}}^k],\;{\bar {\mathbf {r}}}=[{\bar {r}}^k].
\end{equation}
%где $m$-мерные векторы ${\bar {\mathbf{z}}}=[{\bar {z}}^k]$ и ${\bar {\mathbf {r}}}=[{\bar {r}}^k]$ состоят из математических ожиданий, заданных в предположениях {\bf A2.d, A2.e}.
%Заметим, что в случае сбалансированной топологии графа ${\cal G}_{B_{av}}$ $\mathbf x_0^{\star} = \frac{1}{n}\sum_{i=1}^n \mathbf x_0^{i}$.

\noindent{\it Дифференцированный консенсус.}
Рассмотрим векторы ${\bar{\mathbf{X}}}^{\star}_t \in {\mathrm R}^{\bar n}, \; t=0,1, \ldots$, состоящие из $ \mathbf{1}_{n} \otimes \mathbf x_t^{\star}, \mathbf{1}_{n}  \otimes  \mathbf  x_{t-1}^{\star}, \ldots,  \mathbf{1}_{n} \otimes  \mathbf x_{t-\bar s}^{\star}$.

\textit{Теорема 1}:
%\begin{theorem}
\label{thm}
Если для систем с обратными связями~(\ref{Nat_12}) и~(\ref{aver_syst}) выполнены предположения {\bf A1--A3} и выполнены стоимостные ограничения, {\bf то} справедливо следующее неравенство:
%\begin{equation}
% \label{Nat_T1}%\nonumber
$
 { \mathrm{E}}||{\bar{\mathbf{X}}}_{t} - {\bar{\mathbf{X}}}^{\star}_{t}||^{2} \leq \frac{\Delta}{\gamma \delta} + (1-\gamma \delta)^{t}\left(||{\bar{\mathbf{X}}}_{0} - {\bar{\mathbf{X}}}^{\star}_{0}||^{2} - \frac{\Delta}{\gamma \delta}  \right),
 $
% \end{equation}
где
$
\Delta =  2m\sigma_w^2 \gamma^2 (n^2\sigma_{b}^2 + ||B_{av}||^2) + n \sum_{k=1}^m (\sigma_{z,k}^2+ (1-P_k)^2), $
то есть, если $\mathrm{E}||{\bar{\mathbf{X}}}_{0} - {\bar{\mathbf{X}}}^{\star}_{0}||^{2} <\infty, $ то асимптотический среднеквадратичный $\varepsilon$-консенсус в~(\ref{Nat_12}) достигается с
$
\varepsilon \leq \frac{\Delta}{\gamma \delta}.
$
%\end{theorem}


\textit{Теорема 2:}
%\begin{theorem}
{\bf Если} выполнены предположения {\bf A1--A3} и стоимостные ограничения {\bf то} оптимальные значения шагов $\gamma^{\star}_k$, $k=1,\ldots,m$, для каждого протокола из~(\ref{Nat_7}) могут быть получены из следующих формул:
%family of protocols~

\begin{equation}
\label{optimal_step_size}
{\gamma_k ^ {\star}} = - \frac{S_k}{H^k} \Delta^k+\sqrt{\frac{S_k^2}{H^{k 2}} \Delta^{k 2}+\frac{S_k}{H^k}}
\end{equation}
где
$\Delta^k = \frac{Re(\lambda_{\max}(Q))}{R}$.
%\end{theorem}
%\end{theorem}


\section{Заключение}
В статье рассматривается задача достижения дифференцированного консенсуса в мультиагентной сети, обрабатывающей задания разных классов.
В работе предложена рандомизированная управляющая стратегия для достижения дифференцированного консенсуса в мультиагентной сети в условиях переменной структуры связей, помех, задержек при передаче информации по каналам связи, при наличии стоимостных ограничений на использование связей внутри сети. Предложенная стратегия оптимизируется по величине размера шага алгоритма управления. При неизвестных априори параметрах системы для выбора оптимальной величины размера шага возможно использование рандомизированного алгоритма типа SPSA.

\renewcommand\refname{Литература}
\begin{thebibliography}{11}

\bibitem{vest15}
Амелина Н.О., Иванский Ю.В.
Задача достижения дифференцированного консенсуса при стоимостных ограничениях //
Вестник Санкт-Петербургского университета. Серия 1. Математика. Механика. Астрономия. 2015. Т. 2. \No 4. С. 495--506.

\bibitem{kio15}
Ерофеева В.А., Иванский Ю.В., Кияев В.И.
Управление роем динамических объектов на базе мультиагентного подхода //
Компьютерные инструменты в образовании, вып. 6, 2015, С. 34--42.



\bibitem{msc14} 
Amelina N., Granichin O., Granichina O., Ivanskiy Y., and Jiang Y.
Differentiated consensuses in a stochastic network with priorities // 
Proc. of 2014 IEEE Multi-conference on Systems and Control, October 8--10, 2014, Antibes/Nice, France, pp. 264--269.

\bibitem{ecc15} 
Amelina N., Granichin O., Granichina O., Ivanskiy Y., and Jiang Y.
Optimal Step-Size of a Local Voting Protocol for Differentiated Consensuses Achievement in a Stochastic Network with Priorities // Proc. of the 14th European Control Conference (ECC'15), Linz, Austria, July 15--17, 2015, pp. 628--633.

\bibitem{micnon15} 
Amelina N., Granichin O., Granichina O., Ivanskiy Y., and Jiang Y.
Local Voting Protocol for Differentiated Consensuses in a Stochastic Network with Priorities  // Proc. of the IFAC Conference on Modelling, Identification and Control of Nonlinear Systems (MICNON'15), Saint-Petersburg, Russia, June 24--26, 2015, pp. 964--969.

\bibitem{msc15} 
Ivanskiy Y., Amelina N., Granichin O., Granichina O., and Jiang Y.
Optimal Step-Size of a Local Voting Protocol for Differentiated Consensuses Achievement in a Stochastic Network with Cost Constraints for Different Priorities // Proc. of the 2015 IEEE Multi-Conference on Systems and Control (MSC'15), September 21--23, 2015, Sydney, Australia, pp. 1367--1372.

\bibitem{codit16} 
Amelin K., Amelina N., Ivanskiy Y., and Jiang Y.
Choice of Step-Size for Consensus Protocol in Changing Conditions via Stochastic Approximation Type Algorithm // 3rd IEEE International Conference on Control, Decision and Information Technologies (CoDIT'16), April 6--8, 2016, Saint Julian's, Malta.


\bibitem{soi9-1}
Амелин К.С., Баклановский М.В., Граничин О.Н., Иванский Ю.В., Корнивец А.Д., Мальковский Н.В., Найданов Д.Г., Шеин Р.Е. 
Адаптивная мультиагентная операционная система реального времени //
Стохастическая оптимизация в информатике. 2013. Т. 9. \No 1. С. 3--16.

\bibitem{soi9-2}
Иванский Ю.В.
Восстановление 3D-образов из скомпрессированных данных УЗИ //
Стохастическая оптимизация в информатике. 2013. Т. 9. \No 1. С. 45--58.

\bibitem{vspu14} 
Амелина Н.О., Корнивец А.Д., Иванский Ю.В., Тюшев К.И.
Применение консенсусного протокола для балансировки загрузки стохастической децентрализованной сети при передаче данных //
В сб.: XII всероссийское совещание по проблемам управления ВСПУ-2014 Институт проблем управления им. В.А. Трапезникова РАН. 2014. С. 8902--8911.

\bibitem{ts14}
Ivanskiy Yu.
Consensus achievement for different task classes in multi-agent network //
В книге: Управление, информация и оптимизация (VI ТМШ) Тезисы докладов Шестой Традиционной всероссийской молодежной летней Школы. Федеральное государственное бюджетное учреждение науки, Институт проблем управления им. В. А. Трапезникова Российской академии наук, Национальный комитет по автоматическому управлению, Лаборатория структурных методов анализа данных в предсказательном моделировании МФТИ; под редакцией Б. Т. Поляка. 2014. С. 31.

\bibitem{soi10}
Иванский Ю.В.
Дифференцированный консенсус в стохастической сети с приоритетами //
Стохастическая оптимизация в информатике. 2014. Т. 10. \No 1. С. 9--29.

\bibitem{soi11}
Иванский Ю.В.
Использование алгоритма типа стохастической аппроксимации для поиска оптимального шага протокола локального голосования при достижении дифференцированного консенсуса в мультиагентной сети со стоимостными ограничениями на топологию //
Стохастическая оптимизация в информатике. 2015. Т. 11. \No 3. С. 18--38.

\end{thebibliography}

\end{document}
